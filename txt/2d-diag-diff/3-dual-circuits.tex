%!TEX root = ../../THESIS.tex

\subsection{Differentiating quantum circuits}\label{3-dual-circuits}

In this section, we extend diagrammatic differentiation to classical-quantum
circuits. These circuit diagrams have two kinds of wires for bits and qubits,
and boxes for pure quantum processes, measurements and preparations.
We interpret these classical-quantum circuits in terms of parametrised matrices,
where the tensor product reorders the indices to keep the classical and quantum
dimensions in order. Borrowing the term from Coecke and Kissinger
\cite{CoeckeKissinger17}, we call these matrices cq-maps.
In this context, diagrammatic derivations correspond to the notion of gradient
recipe for parametrised quantum gates \cite{SchuldEtAl19}.

We first give the definition of parametrised cq-maps which is at the basis
of our Python implementation.
The category $\mathbf{CQMap}_n$ has objects given by pairs of natural numbers
$\text{Ob}(\mathbf{CQMap}_n) = \N \times \N$, where the first and second element
of the pair encode the classical and the quantum dimension of the system respectively.
Arrows $f : (a, b) \to (c, d)$ are given by $a \times b^2 \to c \times d^2$
parametrised complex matrices, i.e. with entries in $\R^n \to \C$.
Composition of cq-maps is given by multiplying their underlying matrices.
Tensor is given on objects by pointwise multiplication and on arrows by the
following diagram in $\mathbf{Mat}_{\R^n \to \C}$:
$$\tikzfig{img/diag-diff/3-1-tensor}$$

Each pure map $f : a \to b$ in $\mathbf{Mat}_{\R^n \to \C}$ embeds as a
cq-map $(1, a) \to (1, b)$ by ``doubling'', i.e. tensoring with its complex
conjugate $f \mapsto \bar{f} \otimes f$.
Note that doubling is faithful up to a global phase.
For each dimension $a \in \N$, there are distinguished cq-maps
$M_a : (1, a) \to (a, 1), \s E_a : (a, 1) \to (1, a)$ for measurement and
preparation in the computational basis with matrices given by
$M_a = \sum_{i < a} \ket{i} \bra{i, i}$
and $E_a = \sum_{i < a} \ket{i, i} \bra{i}$.
The sum of two cq-maps is given by entrywise addition of their underlying
matrix. Note that doubling does not preserve sums,
i.e. $\overline{(\sum_i f_i)} \otimes (\sum_i f_i)
\neq \sum_i (\overline{f_i} \otimes f_i)$.
In quantum mechanical terms, this corresponds to the distinction between
quantum superposition and probabilistic mixing.

\begin{remark}
The cq-maps we have defined here differ from \cite{CoeckeKissinger17} in two
minor ways. First, we take the algebraic conjugate rather than the diagrammatic
conjugate, i.e. we take $\overline{f \otimes g}
= \overline{f} \otimes \overline{g} \neq \overline{g} \otimes \overline{f}$.
This is just a choice of convention that makes numerical computation easier.
Second, our category $\mathbf{CQMap}$ contains matrices that have no physical
interpretation, e.g. we do not ask for complete positivity. This can be fixed
by considering the subcategory in the image of the interpretation functor
defined below.
\end{remark}

Take a monoidal signature $\Sigma$ with one object $\Sigma_0 = \{ q \}$
interpreted as a qubit, and boxes interpreted as pure quantum processes with
$n$ parameters. That is, we fix a parametrised interpretation functor $[\![-]\!]
: \mathbf{C}_\Sigma \to \mathbf{Mat}_{\R^n \to \C}$ with $[\![q]\!] = 2$.
This could be the signatures for parametrised or algebraic ZX
from the previous section, or any universal quantum gate set plus boxes
for scalars, bras and kets.
We define an extended signature $cq(\Sigma) \supset \Sigma$ with two objects
$cq(\Sigma)_0 = \{ c, q \}$ interpreted as bit and qubit respectively.
Boxes are given by $cq(\Sigma)_1 =
\{ \hat{f} : q^{\otimes a} \to q^{\otimes b} \ \vert \ f \in \Sigma_1 \}
+ \{ M : q \to c, \s E : c \to q \}$.
Let $\mathbf{C}_{cq(\Sigma)}$ be the free monoidal category it generates,
i.e. arrows are classical-quantum circuits.
Their interpretation is given by a monoidal functor
$[\![-]\!] : \mathbf{C}_{cq(\Sigma)} \to \mathbf{CQMap}_n$ with
$[\![c]\!] = (2, 1)$ and $[\![q]\!] = (1, 2)$ on objects.
On arrows we define $[\![M]\!] = M_2$, $[\![E]\!] = E_2$ and
$[\![\hat{f}]\!] = \overline{[\![f]\!]} \otimes [\![f]\!]$.
We write $cq(\mathbf{ZX}_n)$ for the category of classical-quantum circuits
with parametrised ZX diagrams as pure processes.

Let $\mathbf{C}_{cq(\Sigma)}^+$ be the free monoidal category with sums,
i.e. arrows are bags of circuits.
Again, we want to find a diagrammatic derivation
$\partial : \mathbf{C}_{cq(\Sigma)}^+ \to \D[\mathbf{C}_{cq(\Sigma)}^+]$
which commutes with the interpretation, i.e. such that
$[\![\partial \hat{f}]\!] = \partial [\![\hat{f}]\!] =
\partial \big( \overline{[\![f]\!]} \otimes [\![f]\!] \big)$
for all pure maps $f \in \Sigma_1$.
Note that a diagrammatic derivation for pure processes in
$\mathbf{C}_{\Sigma}^+$ does not in general lift to one for classical-quantum
circuits in $\mathbf{C}_{\Sigma}$. Indeed, using the product rule we get
$\partial \big( \overline{[\![f]\!]} \otimes [\![f]\!] \big)
\s = \s \partial \overline{[\![f]\!]} \otimes [\![f]\!]
\ + \ \overline{[\![f]\!]} \otimes \partial [\![f]\!]
\s \neq \s \overline{[\![\partial f]\!]} \otimes [\![\partial f]\!]$.

Hence we need equations, called gradient recipes, to rewrite the gradient of a
pure map $\partial [\![\hat{f}]\!]$ as the pure map of a gradient
$[\![\partial \hat{f}]\!]$.
In the special case of Hermitian operators with at most two unique eigenvalues,
gradient recipes are given by the parameter-shift rule. In the general case
where the parameter-shift rule does not apply, gradient recipes require the
introduction of an ancilla qubit.

\begin{theorem}[Schuld et al.]
For a one-parameter unitary group $f$ with
$[\![f(\theta)]\!] = \exp (i \theta H)$, if $H$ has at most two eigenvalues
$\pm r$, then there is a shift $s \in [0, 2 \pi)$ such that
$[\![r\big(f(\theta + s) - f(\theta - s)\big)]\!] = \partial [\![f(\theta)]\!]$.
\end{theorem}

\begin{proof}
The shift is given by $s = \frac{\pi}{4 r}$, see the Taylor expansion given in
\cite[Theorem 1]{SchuldEtAl19}.
\end{proof}

\begin{corollary}
Classical-quantum circuits $cq(\mathbf{ZX}_n)$ with parametrised ZX diagrams as
pure processes admit diagrammatic differentiation.
\end{corollary}

\begin{proof}
The $Z$ rotation has eigenvalues $\pm 1$, hence the spiders with two legs have
diagrammatic differentiation given by the parameter-shift rule:

\ctikzfig{img/diag-diff/3-2-param-shift}

As for theorem~\ref{theorem-zx-diag-diff}, this extends to
arbitrary-many legs using spider fusion.
\end{proof}

\begin{remark}
All scalars in $cq(\mathbf{ZX}_n)$ are non-negative real numbers. Thus in order
to encode the substraction of the parameter shift-rule diagrammatically, we
need either to consider formal sums with minus signs (a.k.a. enrichment in
Abelian groups) or simply to extend the signature with the $-1$ scalar.
\end{remark}

\begin{example}
The quantum enhanced feature spaces of \cite{HavlicekEtAl19} are parametrised
classical-quantum circuits.
The quantum classifier can be drawn as a diagram:

\ctikzfig{img/diag-diff/3-3-quantum-enhanced}

where $U(\vec{x})$ depends on the input, $W(\vec{\theta})$ depends on the
trainable parameters and $f$ is a fixed Boolean function encoded as a linear map.
\end{example}
