%!TEX root = ../../THESIS.tex

\subsection{Hypergraph categories \& wire splitting} \label{subsection:hypergraph}

With compact closed and tortile categories, we have removed both the progressivity and the planarity assumptions: wires can bend and cross.
With \emph{hypergraph categories} we remove the assumption that diagrams are graphs: wires can split and merge, they need not be homeomorphic to an open interval.
A hypergraph category is a symmetric category with \emph{coherent special commutative spiders}, let's spell out what this means.

An object $x$ in a monoidal category $C$ has \emph{spiders} with \emph{phases} in a monoid $(\Phi, +, 0)$ if it comes equipped with a family of arrows $\spider_{\phi, a, b}(x) : x^a \to x^b$ for every phase $\phi \in \Phi$ and pair of natural numbers $a, b \in \N$, such that the following \emph{spider fusion} equation holds for all $a, b, c, d, n \in \N$.
$$\spider_{\phi, a, c + n + 1}(x) \otimes x^b
\ \fcmp \ x^c \otimes \spider_{\phi', c + n + 1, d}(x)
\s = \s \spider_{\phi + \phi', a + b, c + d}(x)$$
We also require that our spiders satisfy the \emph{special} condition $\spider_{0, 1, 1}(x) = \id(x)$.
Spiders owe their name to their arachnomorphic drawing, for example $\spider_{\phi, 2, 6}$ is drawn as a node (the head, labeled by its phase when it's non-zero) and its wires (the eight legs of the spider, two of them menacing us):
\ctikzfig{img/hypergraph/spider}
Once drawn, the spider fusion equation has the intuitive graphical meaning that if one or more legs of two spiders touch, they fuse and add up their phase.
\ctikzfig{img/hypergraph/spider-fusion}
From spider fusion, we can deduce the following properties:
\begin{itemize}
\item $\merge(x) = \spider_{0, 2, 1}(x)$ and $\unit(x) = \spider_{0, 0, 1}(x)$ form a monoid,
\begin{center}
\tikzfig{img/hypergraph/assoc}
\hfill
\tikzfig{img/hypergraph/unit}
\end{center}
\item $\ttsplit(x) = \spider_{0, 1, 2}(x)$ and $\counit(x) = \spider_{0, 1, 0}(x)$ form a comonoid,
\begin{center}
\tikzfig{img/hypergraph/coassoc}
\hfill
\tikzfig{img/hypergraph/counit}
\end{center}
\item $\ttsplit(x) \fcmp \merge(x) = \id(x)$, called the \emph{special} condition,
\ctikzfig{img/hypergraph/special}
\item $\merge(x) \otimes x \fcmp x \otimes \ttsplit(x) \s = \s x \otimes \merge(x) \fcmp \ttsplit(x) \otimes x$, called the \emph{Frobenius law}.
\ctikzfig{img/hypergraph/frobenius}
\end{itemize}
In fact, when the phases are trivial $\Phi = \{ 0 \}$ these four axioms are sufficient to deduce spider fusion, spiders are also called \emph{special Frobenius algebras}.
Indeed, given a monoid $\merge(x) : x \otimes x \to x, \s \unit(x) : 1 \to x$ and a comonoid $\ttsplit(x) : x \to x \otimes x, \s \counit(x) : x \to 1$ subject to the Frobenius law, we can construct $\spider_{a, b}(x) : x^a \to x^b$ by induction on the number of legs.
The base case is given by the special condition $\spider_{1, 1}(x) = \id(x)$.
Then we define spiders with $a \in \N$ input legs for $a \neq 1$:
\begin{itemize}
\item $\spider_{0, b}(x) \s = \s \unit(x) \fcmp \spider_{1, b}(x)$,
\item $\spider_{a + 2, b}(x) \s = \s \merge(x) \otimes x^a \ \fcmp \ \spider_{a + 1, b}(x)$,
\end{itemize}
\begin{center}
\tikzfig{img/hypergraph/induction-base}
\hfill
\tikzfig{img/hypergraph/induction-step}
\end{center}
Finally we define spiders with one input leg by induction on the output legs $b \in \N$:
\begin{itemize}
\item $\spider_{1, 0}(x) \s = \s \counit(x)$,
\item $\spider_{1, b + 2}(x) \s = \s \spider_{1, b + 1}(x) \ \fcmp \ \ttsplit(x) \otimes x^b$.
\ctikzfig{img/hypergraph/induction-step-one-legged}
\end{itemize}
One can show that this satisfies the spider fusion law, again by induction on the legs~\cite[Lemma 5.20]{HeunenVicary19a}.
In this way, we can construct an infinite family of spiders from just the four boxes $\merge(x), \unit(x), \ttsplit(x), \counit(x)$ and a finite set of equations: a spider is nothing but a big multiplication followed by a big co-multiplication.
As for the phases, we can recover them from a family of \emph{phase shifts} $\{ \shift_\phi(x) : x \to x \}_{\phi \in \Phi}$ such that:
\begin{itemize}
\item $\shift_-(x)$ is a monoid homomorphism $\Phi \to C(x, x)$, i.e. $\shift_0(x) = \id(x)$ and $\shift_{\phi}(x) \fcmp \shift_{\phi'} = \shift_{\phi + \phi'}(x)$,
\ctikzfig{img/hypergraph/phase-hom}
\item phase shifts commute with the product, $\shift_\phi(x) \otimes x \ \fcmp \ \merge(x)
\ = \ \merge(x) \fcmp \shift_\phi(x)
\ = \ x \otimes \shift_\phi(x) \ \fcmp \ \merge(x)$,
\ctikzfig{img/hypergraph/phase-commute-product}
\item phase shifts commute with the coproduct, $\ttsplit(x) \ \fcmp \ \shift_\phi(x) \otimes x
\ = \ \shift_\phi(x) \fcmp \ttsplit(x)
\ = \ x \otimes \ttsplit(x) \ \fcmp \ \shift_\phi(x)$.
\ctikzfig{img/hypergraph/phase-commute-coproduct}
\end{itemize}
We can then define $\spider_{\phi, a, b}(x) = \spider_{a, 1}(x) \fcmp \shift_\phi(x) \fcmp \spider_{1, b}(x)$ and check that indeed, spiders fuse up to addition of their phase.
Thus when the monoid is finite, we get a finite number of boxes and equations, i.e. a finite presentation of the spiders.
In fact instead of taking it as data, we could have equivalently defined the monoid of phases $\Phi$ as the set of endomorphisms $x \to x$ that satisfy the last two conditions.

\begin{remark}
Given any Frobenius algebra on an object $x$, we can show that $x$ is its own left and right adjoint.
Indeed, take $\ttcup(x) = \unit(x) \fcmp \ttsplit(x)$ and $\ttcap(x) = \merge(x) \fcmp \counit(x)$, then the Frobenius law and the (co)unit law of the (co)monoid implies the snake equations.
Thus, a category with (not-necessarily special) spiders on every object is automatically a pivotal category.

\ctikzfig{img/hypergraph/spider-implies-snake}
\end{remark}

\begin{example}
In any pivotal category, there is a Frobenius algebra for every object of the form $x^\star \otimes x$ given by:
\begin{itemize}
\item $\merge(x^\star \otimes x) = x^\star \otimes \ttcup(x^\star) \otimes x$ and $\unit(x) = \ttcap(x)$,
\item $\ttsplit(x^\star \otimes x) = x^\star \otimes \ttcap(x) \otimes x$ and $\counit(x) = \ttcup(x^\star)$.
\end{itemize}
\ctikzfig{img/hypergraph/pair-of-pants}
Due to the drawing of its comonoid, this is called the \emph{pair of pants} algebra.
The special condition requires the dimension of the system $x$ to be the unit, i.e. the circle is equal to the empty diagram.
Non-special Frobenius algebras can still be drawn as spiders, they satisfy a modifed version of spider fusion where we keep track of the number of circles, i.e. the number of splits followed by a merge.
We can extend our inductive definition so that all the circles are in between the product and coproduct, see~\cite[Theorem 5.21]{HeunenVicary19a}.
\end{example}

\begin{example}\label{example:tensor-spider}
The category $\mathbf{Tensor}_\S$ has spiders for every dimension $n \in \N$ with phases in any submonoid of $\phi \in (\S, \times, 1)^n$.
They are given by $\spider_{\phi, a, b}(n) = \sum_{i \leq n} \phi_i \ket{i}^{\otimes a} \bra{i}^{\otimes b}$ where $\ket{i}$ ($\bra{i}$) is the $i$-th basis row (column) vector.

\begin{minted}{python}
class Tensor:
    ...
    @classmethod
    def spider(cls, a: int, b: int, n: int, phase=None) -> Tensor:
        phase = phase or n * [1]
        inside = [[sum(phase)]] if not a and not b\
            else [[phase[xs[0]] for xs in itertools.product(*b * [range(n)])
                   if all(x == xs[0] for x in xs)]]\
            if not a else cls.spider([], a + b, n).inside
        return cls(inside, dom=a * [n], cod=b * [n])
\end{minted}

When $\S$ is a field, we can divide every $\phi_i$ by $\phi_0$, or equivalently require that $\phi_0 = 1$.
Indeed, we can represent any spider with $\phi_0 \neq 1$ as a spider with $\phi_0 = 1$ multiplied by the scalar $\phi_0$, which is called a \emph{global phase}.
When $\S = \C$ and $n = 2$, we usually take the monoid of phases to be the unit circle and write it in terms of addition of angles.
\end{example}

\begin{example}\label{example:circuit-spider}
In the category $\mathbf{Circ}$ of quantum circuits, if we allow post-selected measurements then we can construct spiders with the unit circle as phases.
The spiders with no inputs legs are called the (generalised) GHZ states:
$$
\spider_{\alpha, 0, b} = \ket{0}^{\otimes b} + e^{i \alpha} \ket{1}^{\otimes b}
$$
Note that we need to scale by $\frac{1}{\sqrt{2}}$ to make this a normalised quantum state.
The spiders with $a > 0$ input legs can be thought of as measuring $a$ qubits, post-selecting on all of them giving the same result and then preparing $b$ copies of this result.
The evaluation functor $\mathbf{Circ} \to \mathbf{Tensor}_\C$ sends spiders to spiders.
\end{example}

Spiders allow us to draw diagrams where wires can split and merge, connecting an arbitrary number of boxes.
The PRO of Frobenius algebras (without the special condition), i.e. diagrams with only spider boxes, defines a notion of ``well-behaved'' 1d subspaces of the plane, up to continuous deformation.
Indeed, it is equivalent to the category of \emph{planar thick tangles}~\cite{Lauda05}.
Intuitively, planar thick tangles can be thought of as planar wires with a width, i.e. that we can draw with pens or pixels.
The inductive definition of spiders in terms of monoids and comonoids has the topological interpretation that any wire can be deformed so that all its singular points (i.e. where the wire crosses itself) are binary splits and merges.
The special condition has the non-topological consequence that we can contract the holes in the wires, splitting a wire then merging it back does nothing.

If the monoidal category $C$ is braided, we can remove the planarity assumption and define \emph{commutative spiders} as those where the monoid and comonoid are commutative, i.e.
\begin{align*}
\spider_{\phi, a + b, c + d}(x) \ \fcmp \ B(x^c, x^d)
\s &= \s \spider_{\phi, a + b, c + d}(x) \\
\s &= \s
B(x^a, x^b) \ \fcmp \ \spider_{\phi, a + b, c + d}(x)
\end{align*}
\ctikzfig{img/hypergraph/co-commutativity}
Together with spider fusion, this implies that the monoid of phases is also commutative.
The PROB of commutative Frobenius algebras (without the special condition), i.e. diagrams with only spiders and braids, defines a notion of ``well-behaved'' 1d subspaces of 3d space, up to continuous deformation.
When the category is furthermore symmetric, the PROP of commutative spiders defines a notion of ``well-behaved'' 1d spaces up to diffeomorphism, or equivalently 1d subspaces of 4d space, i.e. one where wires can pass through each other and all knots untie.
It is equivalent to the category of two-dimensional \emph{cobordisms}~\cite{Abrams96}, i.e. oriented 2d manifolds with a disjoint union of circles as boundary.
Intuitively, a 2d cobordism can be thought of as a (non-planar) wire with a width, i.e. one that we can draw.

When $C$ is braided, we can also give an inductive definition of spiders for tensors.
Indeed, given the spiders for $x$ and $y$ we can construct the following comonoid:
\begin{itemize}
\item $\spider_{1, 0}(x \otimes y) \s = \s \spider_{1, 0}(x) \otimes \spider_{1, 0}(y)$,
\ctikzfig{img/hypergraph/coherence-unit}
\item $\spider_{1, 2}(x \otimes y) \s = \s \spider_{1, 2}(x) \otimes \spider_{1, 2}(y) \ \fcmp \ x \otimes S(x, y) \otimes y$
\ctikzfig{img/hypergraph/coherence-product}
\end{itemize}
and construct a monoid in a symmetric way, then show that they satisfy the spider fusion equations for $x \otimes y$.
We can also show that the identity of the unit defines a family of spiders, i.e. $\spider_{a, b}(1) = \id(1)$.
If we take them as axioms rather than definitions, these are called the \emph{coherence conditions} for spiders.

Thus we get to our definition: a \emph{hypergraph category} is a symmetric category with coherent special commutative spiders on each object.
We can take the data to be that of a foo-monoidal category $C$ together with a function $\spider : \N \times \N \times C_0 \to C_1$ or equivalenty, with four functions $\merge, \unit, \ttsplit, \counit : C_0 \to C_1$.
Once we fix the spiders for generating objects, we get spiders for any type (i.e. list of objects).
A hypergraph functor is a symmetric functor $F : C \to D$ between hypergraph categories such that $F \fcmp \spider_{a, b} = \spider_{a, b} \fcmp F$.
Thus we get a category $\mathbf{HypCat}$ with a forgetful functor $U : \mathbf{HypCat} \to \mathbf{MonSig}$.
Its left adjoint $F^H : \mathbf{MonSig} \to \mathbf{HypCat}$ is defined as a quotient $F^S(\Sigma^H) / R$ of the free symmetric category generated by $\Sigma^H = \Sigma \spider$ and the relation $R$ given by the equations for commutative spiders.
Equivalently, we can take $\Sigma^H = \bigcup \{ \Sigma, \merge, \unit, \ttsplit, \counit \}$ and $R$ given by the equations for special commutative Frobenius algebras.
A \emph{$\dagger$-hypergraph category} is a $\dagger$-symmetric category (i.e. the swaps are unitaries) where the dagger is a hypergraph functor.
We also require that the monoid of phases is in fact a group with the dagger as inverse or equivalently, that phase shifts are unitaries.

\begin{example}
For evert commutative rig $\S$, the category $\mathbf{Tensor}_\S$ is $\dagger$-hypergraph with the transpose as dagger.
Arguably, special commutative Frobenius algebras were first defined by Peirce~\cite{Peirce06} with their interpretation in the category of relations, or equivalently $\mathbf{Tensor}_\B$.
Indeed, they correspond to what Peirce calls \emph{lines of identity}: they express in two dimensions what one-dimensional first-order logic would express with equality symbols.
For example, take a binary predicate encoded as a box $p : 1 \to x^2$ (interpreted as the formula $\exists \ a \cdot \exists \ b \cdot p(a, b)$) then the diagram $p \fcmp \merge(x)$ is interpreted as the formula $\exists \ a \cdot \exists \ b \cdot p(a, b) \land a = b$ or equivalently $\exists \ a \cdot p(a, a)$.
Thus, every first-order logic formula can be written as a diagram with boxes for predicates, spiders for identity and bubbles for negation.
The equivalence of formulae can be defined as a quotient of a free hypergraph category with bubbles, i.e. all the rules of first-order logic can be given in terms of diagrams.
\end{example}

\begin{example}
The category of complex tensors $\mathbf{Tensor}_\C$ is $\dagger$-hypergraph with the spiders given in example~\ref{example:tensor-spider}.
Any unitary matrix $U : n \to n$ defines another family of spiders $U^{\otimes a} \fcmp \spider_{\phi, a, b}(n) \fcmp (U^\dagger)^{\otimes b}$.
In fact, every unitary arises in this way, see Heunen and Vicary~\cite[Corollary 5.32]{HeunenVicary19a}.
Thus, the axioms for spiders allow us to define any orthonormal basis without ever mentioning basis vectors: they are merely the states $v : 1 \to n$ for which the comonoid is natural, i.e. $v \fcmp \ttsplit(x) = v \otimes v$ and $v \fcmp \counit(x) = \id(1)$.
\end{example}

\begin{example}
The category $\mathbf{Circ}$ is $\dagger$-hypergraph with the spiders defined in example~\ref{example:circuit-spider}, the evaluation functor $\mathbf{Circ} \to \mathbf{Tensor}_\C$ is a $\dagger$-hypergraph functor.
\end{example}

DisCoPy implements spiders for types of length one (i.e. generating objects) as a subclass of \py{Box} and spiders for arbitrary types as a method \py{Diagram.spiders}.

\begin{python}
{\normalfont Implementation of $\dagger$-hypergraph categories and functors.}

\begin{minted}{python}
class Spider(Box):
    def __init__(self, a: int, b: int, x: Ty, phase=None):
        assert len(x) == 1
        self.object, self.phase = x, phase or 0
        name = "Spider({})".format(', '.join(map(str, (a, b, x, phase))))
        super().__init__(name, dom=x ** a, cod=x ** b)

    def dagger(self):
        a, b, x = len(self.cod), len(self.dom), self.object
        phase = None if self.phase is None else -self.phase
        return Spider(a, b, x, phase)

def coherence(factory):
    @classmethod
    def method(cls, a: int, b: int, x: Ty, phase=None) -> Diagram:
        if len(x) == 0 and phase is None: return cls.id(x)
        if len(x) == 1: return factory(a, b, x, phase)
        if phase is not None:  # Coherence for phase shifters.
            shift = cls.tensor(*[factory(1, 1, obj, phase) for obj in x])
            return method(cls, a, 1, x) >> shift >> method(cls, 1, b, x)
        if (a, b) in [(1, 0), (0, 1)]: # Coherence for (co)units.
            return cls.tensor(*[factory(a, b, obj) for obj in x])
        if (co, a, b) == [(True, 1, 2), (False, 2, 1)]:  # Coherence for (co)products.
            product_or_co = factory(a, b, x[0]) @ method(cls, a, b, x[1:])
            braid = x[0] @ cls.braid(x[0], x[1:]) @ x[1:]
            return product_or_co >> braid if co else braid >> product_or_co
        if a == 1:  # We can now assume b > 2.
            return method(cls, 1, b - 1, x) >> method(cls, 1, 2, x) @ (x ** (b - 2))
        if b == 1:  # We can now assume a > 2.
            return method(cls, 2, 1, x) @ (x ** (a - 2)) >> method(cls, a - 1, 1, x)
        return method(cls, a, 1, x) >> method(cls, 1, b, x)

Diagram.spiders = coherence(Spider)
Diagram.cups = nesting(lambda x, _: Spider(0, 2, x))
Diagram.caps = nesting(lambda x, _: Spider(2, 0, x))

class Functor(braided.Functor):
    def __call__(self, other):
        if isinstance(other, Spider):
            a, b, x, phase = len(other.dom), len(other.cod), other.object, other.phase
            return self.cod.ar.spiders(a, b, self(x), phase)
        return super().__call__(other)
\end{minted}
\end{python}

\begin{example}
We can now extend example~\ref{example:monoidal-formula} to arbitrary formulae of first-order logic.
Every variable that appears exactly twice is encoded as a wire (possibly with cups and caps), every variable that appears $n \neq 2$ is encoded as an $n$-legged spider.
For example, the formula $\forall c \ \forall o \ O(c, o) \land R(c) \land C(c) \implies U(c, o)$ (interpreted as ``every object of a rigid cartesian category is also its unit'') can be encoded as a diagram with a wire for $o$ and a four-legged spider for $c$.

\begin{minted}{python}
class Formula(Diagram):
    cut = lambda self: self.bubble(name="_not")

class Predicate(Box, Formula):
    def __init__(self, name, dom): Box.__init__(self, name, Ty(), dom)

def model(size: dict[Ty, int], data: dict[Predicate, list[bool]]):
    return Functor(ob=size, ar={p: [data[p]] for p in data},
                   dom=Category(Ty, Formula), cod=Category(list[int], Tensor[bool]))

objects, categories = Ty('o'), Ty('c')
has_object, has_unit = [Predicate(name, categories @ objects) for name in "OU"]
is_rigid, is_cartesian = [Predicate(name, categories) for name in "RC"]

rigid_cartesian_implies_trivial = (
    has_object >> Formula.spiders(1, 3, categories) @ objects
    >> (is_rigid @ is_cartesian @ has_unit.cut()).dagger()).cut()

size = {objects: 2, categories: 2}
predicate_values = itertools.product(*size[categories] * [[0, 1]])
relation_values = itertools.product(*size[categories] * size[o  bjects] * [[0, 1]])

for O, U, R, C in itertools.product(2 * [predicate_values] + 2 * [relation_values]):
    F = model(size, {has_object: O, has_unit: U, is_rigid: R, is_cartesian: C})
    is_rigid_cartesian_and_has_object = lambda i, j:\
        F(has_object)[i, j] and F(is_rigid)[i] and F(is_cartesian)[i]
    assert F(rigid_cartesian_implies_trivial) == all(
        not is_rigid_cartesian_and_has_object(i, j) or F(has_unit)[i, j]
        for i in range(size[categories]) for j in range(size[objects]))

rigid_cartesian_implies_trivial.draw()
\end{minted}

\ctikzfig{img/hypergraph/rigid-cartesian-implies-trivial}
\end{example}

The equality of hypergraph diagrams reduces to hypergraph isomorphism, it will be discussed in section~\ref{subsection:hypergraph-vs-premonoidal}.
The equality of non-commutative spiders is not implemented yet, spider fusion would be a natural extension of the snake removal algorithm for rigid diagrams: we find pairs fusable spiders then apply interchangers to make them adjacent.
The possible obstructions are more serious for spiders than for cups and caps however, for example consider the diagram $\spider_{0, 3}(x) \ \fcmp \ x \otimes f \otimes g \ \fcmp \ \spider_{3, 0}(x)$.
The two three-legged spiders want to fuse but the boxes $f$ and $g$ stand on the way, the best we can do is to bend their output wires with two cups and get a four-legged spider $\spider_{0, 4}(x) \ \fcmp \ x \otimes f \otimes g \otimes x \ \fcmp \ \ttcup(x) \otimes \ttcup(x)$.

\ctikzfig{img/hypergraph/obstruction}
