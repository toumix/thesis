%!TEX root = ../THESIS.tex

\documentclass[a4paper,twoside]{src/ociamthesis}

\usepackage[utf8]{inputenc}
\usepackage[english,greek]{babel}
% \usepackage{breakurl}

\usepackage{hyperref}
%WTF https://myjian.wordpress.com/2014/01/17/note-break-long-urls-with-xelatex/
\renewcommand{\UrlBreaks}{\do\/\do\-\do\_\do\.\do\a\do\b\do\c\do\d\do\e\do\f\do\g\do\h\do\i\do\j\do\k\do\l\do\m\do\n\do\o\do\p\do\q\do\r\do\s\do\t\do\u\do\v\do\w\do\x\do\y\do\z\do\A\do\B\do\C\do\D\do\E\do\F\do\G\do\H\do\I\do\J\do\K\do\L\do\M\do\N\do\O\do\P\do\Q\do\R\do\S\do\T\do\U\do\V\do\W\do\X\do\Y\do\Z}


\usepackage[
    style=alphabetic,
    sorting=nyt,
    backend=biber,
    maxbibnames=99,
    doi=true,
    isbn=false,
    url=true
]{biblatex}
\renewcommand*{\bibfont}{\raggedright\small}

% print url if no doi
\DeclareSourcemap{
  \maps[datatype=bibtex]{
    \map[overwrite]{
      \step[fieldsource=doi, final]
      \step[fieldset=url, null]
      \step[fieldset=urldate, null]
      \step[fieldset=eprint, null]
    }
  }
}
% print url if no arXiv
\DeclareSourcemap{
  \maps[datatype=bibtex]{
    \map[overwrite]{
      \step[fieldsource=eprint, final]
      \step[fieldset=url, null]
      \step[fieldset=urldate, null]
    }
  }
}

% https://tex.stackexchange.com/questions/163774/biblatex-print-bibliography-for-a-single-entry-within-an-enumeration
\DeclareBibliographyCategory{enumpapers}

\newcommand{\enumcite}[1]{%
  \nocite{#1}
  \addtocategory{enumpapers}{#1}%
  \defbibcheck{key#1}{
    \iffieldequalstr{entrykey}{#1}
      {}
      {\skipentry}}%
  \printbibliography[heading=none,check=key#1]%
}

\usepackage{amsthm, amssymb, amsmath}

\usepackage{fontspec}
\setmonofont{Menlo}[Scale=MatchLowercase]

\usepackage{setspace}
\renewcommand{\baselinestretch}{1.25}

\usepackage{minted}
\usemintedstyle{lovelace}
\setminted{frame=lines,fontseries=mono,fontsize=\footnotesize}
\setmintedinline{fontsize=auto}
\BeforeBeginEnvironment{minted}{\vspace{-16pt}}
\AfterEndEnvironment{minted}{\vspace{-8pt}}

\newcommand{\py}[1]{{\color{brown}\mintinline{python}{#1}}}

\newcommand{\inputpython}[1]{
{\vspace{-10pt} \footnotesize \inputminted[baselinestretch=1]{python}{#1}} \vspace{-10pt}}

\usepackage{caption}
\usepackage{subcaption}

\usepackage{graphicx}
\graphicspath{ {./img/} }

\usepackage{tikz}
\usepackage{tikz-cd}
\usepackage{src/tikzit}
\input{src/qcs.tikzstyles}

\usepackage[perpage]{footmisc}

\usepackage{ragged2e}
\usepackage{epigraph}
\setlength{\epigraphwidth}{0.5\textwidth}
\newcommand{\justepigraph}[2]{
\epigraph{\footnotesize \justifying #1}{#2}}



\newcommand{\xto}[1]{\xrightarrow{#1}}
\newcommand{\sub}{\subseteq}
\newcommand{\s}{\enspace}
\newcommand{\eval}[1]{[ \! [ #1 ] \! ]}

\newcommand{\bra}[1]{\langle#1|}
\newcommand{\ket}[1]{|#1\rangle}
\newcommand{\braket}[2]{\langle#1|#2\rangle}

\newcommand{\dom}{\mathtt{dom}}
\newcommand{\cod}{\mathtt{cod}}
\newcommand{\id}{\mathtt{id}}
\newcommand{\then}{\mathtt{then}}
\newcommand{\len}{\mathtt{len}}

\renewcommand{\S}{\mathbb{S}}
\newcommand{\B}{\mathbb{B}}
\newcommand{\N}{\mathbb{N}}
\newcommand{\Z}{\mathbb{Z}}
\newcommand{\R}{\mathbb{R}}
\newcommand{\C}{\mathbb{C}}

\def\fcmp{\mathbin{\raise 0.6ex\hbox{\oalign{\hfil$\scriptscriptstyle      \mathrm{o}$\hfil\cr\hfil$\scriptscriptstyle\mathrm{9}$\hfil}}}}

\newcommand{\py}[1]{{\color{gray}\fbox{\raisebox{0pt}[8pt][2pt]{\color{brown}\normalfont\texttt{\pyth{#1}}}}}}


\newtheorem{definition}{Definition}[section]
\newtheorem{proposition}[definition]{Proposition}
\newtheorem{theorem}[definition]{Theorem}
\newtheorem{conjecture}[definition]{Conjecture}
\newtheorem{lemma}[definition]{Lemma}
\newtheorem{corollary}[definition]{Corollary}
\newtheorem{example}[definition]{Example}
\newtheorem{remark}[definition]{Remark}
\newtheorem{python}[definition]{Listing}


\title{Category Theory for Quantum\\
Natural Language Processing}
\author{Alexis TOUMI}
\college{Wolfson College}

\degree{Doctor of Philosophy}
\degreedate{\today}

\addbibresource{THESIS.bib}

\begin{document}
\begin{romanpages}
\maketitle

%!TEX root = ../../THESIS.tex

\begin{abstract}
This thesis introduces quantum natural language processing (QNLP) models based on a simple yet powerful analogy between computational linguistics and quantum mechanics: grammar as entanglement.
The grammatical structure of text and sentences connects the meaning of words in the same way that entanglement structure connects the states of quantum systems.
Category theory allows to make this language-to-qubit analogy formal: it is a monoidal functor from grammar to vector spaces.
We turn this abstract analogy into a concrete algorithm that translates the grammatical structure onto the architecture of parameterised quantum circuits.
We then use a hybrid classical-quantum algorithm to train the model so that evaluating the circuits computes the meaning of sentences in data-driven tasks.

The implementation of QNLP models motivated the development of DisCoPy (Distributional Compositional Python), the toolkit for applied category theory of which the first chapter gives a comprehensive overview.
String diagrams are the core data structure of DisCoPy, they allow to reason about computation at a high level of abstraction.
We show how they can encode both grammatical structures and quantum circuits, but also logical formulae, neural networks or arbitrary Python code.
Monoidal functors allow to translate these abstract diagrams into concrete computation, interfacing with optimised task-specific libraries.

The second chapter uses DisCopy to implement QNLP models as parameterised functors from grammar to quantum circuits.
It gives a first proof-of-concept for the more general concept of functorial learning: generalising machine learning from functions to functors by learning from diagram-like data.
In order to learn optimal functor parameters via gradient descent, we introduce the notion of diagrammatic differentiation: a graphical calculus for computing the gradients of quantum circuits and parameterised diagrams in general.
\end{abstract}


% \flushbottom
\tableofcontents

% The Roman pages, like the Roman Empire, must come to its inevitable close.
\end{romanpages}

\chapter*{Introduction}
\addcontentsline{toc}{chapter}{Introduction}
\markboth{Introduction}{Introduction}

%!TEX root = ../THESIS.tex

\section*{What are quantum computers good for?}
\addcontentsline{toc}{section}{What are quantum computers good for?}

\justepigraph{
Nature isn't classical, dammit, and if you want to make a simulation of nature, you'd better make it quantum mechanical, and by golly it's a wonderful problem, because it doesn't look so easy.
}{\textit{Simulating Physics with Computers}, Feynman (1981)}

Quantum computers harness the principles of quantum theory such as superposition and entanglement to solve information-processing tasks.
In the last 42 years, quantum computing has gone from theoretical speculations to the implementation of machines that can solve problems beyond what is possible with classical means.
This section will sketch a brief and biased history of the field and of its future challenges.

In 1980, Benioff~\cite{Benioff80} takes the abstract definition of a computer and makes it physical: he designs a quantum mechanical system whose time evolution encodes the computation steps of a given Turing machine.
In retrospect, this may be taken as the first proof that quantum mechanics can simulate classical computers.
The same year, Manin~\cite{Manin80} looks at the opposite direction: he argues that it would take exponential time for a classical computer to simulate a generic quantum system.
Feynman~\cite{Feynman82,Feynman85} comes to the same conclusion and suggests a way to simulate quantum mechanics much more efficiently: building a quantum computer!

So what are quantum computers good for? Feynman's intuition gives us a first, trivial answer: at least quantum computers could simulate quantum mechanics efficiently. Deutsch~\cite{Deutsch85a} makes the question formal by defining quantum Turing machines and the circuit model.
Deutsch and Jozsa~\cite{DeutschJozsa92} design the first quantum algorithm and prove that it solves \emph{some} problem exponentially faster than any classical \emph{deterministic} algorithm.\footnote
{A classical \emph{randomised} algorithm solves the problem in constant time with high probability.}
Simon~\cite{Simon94} improves on their result by designing a problem that a quantum computer can solve exponentially faster than any classical algorithm.
Deutsch-Jozsa and Simon relied on oracles\footnote{An oracle is a black box that allows a Turing machine to solve a certain problem in one step.}
and promises\footnote{The input is promised to satisfy a certain property, which may be hard to check.} and their problems have little practical use.
However, they inspired Shor's algorithm~\cite{Shor94} for prime factorisation and discrete logarithm. These two problems are believed to require exponential time for a classical computer and their hardness is at the basis of the public-key cryptography schemes currently used on the internet.

In 1997, Grover provides another application for quantum computers: ``searching for a needle in a haystack''~\cite{Grover97}.
Formally, given a function $f : X \to \{0, 1\}$ and the promise that there is a unique $x \in X$ with $f(x) = 1$, Grover's algorithm finds $x$ in $O(\sqrt{|X|})$ steps, quadratically faster than the optimal $O(|X|)$ classical algorithm.
Grover's algorithm may be used to brute-force symmetric cryptographic keys twice bigger than what is possible classically~\cite{BernsteinEtAl09}.
It can also be used to obtain quadratic speedups for the exhaustive search involved in the solution of NP-hard problems such as constraint satisfaction~\cite{Ambainis04}.
Independently, Benett et al.~\cite{BennettEtAl97} prove that Grover's algorithm is in fact optimal, adding evidence to the conjecture that quantum computers cannot solve these NP-hard problems in polynomial time.
Chuang et al.~\cite{ChuangEtAl98} give the first experimental demonstration of a quantum algorithm, running Grover's algorithm on two qubits.

Shor's and Grover's discovery of the first real-world applications sparked a considerable interest in quantum computing.
The core of these two algorithms has then been abstracted away in terms of two
subroutines: phase estimation~\cite{Kitaev95} and amplitude
amplification~\cite{BrassardEtAl02}, respectively.
Making use of both these subroutines, the HHL\footnote{Named after its discoverers Harrow, Hassidim and Lloyd.} algorithm~\cite{HarrowEtAl09} tackles one of the most ubiquitous problems in scientific computing: solving systems of linear equations.
Given a matrix $A \in \mathbb{R}^{n \times n}$ and a vector ${b} \in \mathbb{R}^{n}$, we want to find a vector ${x}$ such that $A {x} = {b}$.
Under some assumptions on the sparsity and the condition number of $A$, HHL finds (an approximation of) $x$ in time logarithmic in $n$ when a classical algorithm would take quadratic time simply to read the entries of $A$.
This initiated a new wave of enthusiasm for quantum computing with the promise of exponential speedups for machine learning tasks such as regression~\cite{WiebeEtAl12}, clustering~\cite{LloydEtAl13}, classification~\cite{RebentrostEtAl14}, dimensionality reduction~\cite{LloydEtAl14} and recommendation~\cite{KerenidisPrakash16}.
The narrative is appealing: machine learning is about finding patterns in large amounts of data represented as high-dimensional vectors and tensors, which is precisely what quantum computers are good at.
The argument can be formalised in terms of complexity theory: HHL is \texttt{BQP}-complete\footnote
{A \texttt{BQP}-complete problem is one that is polynomial-time  equivalent to the circuit model, the hardest problem that a quantum computer can solve with bounded error in polynomial time.}
hence if there is an exponential advantage for quantum algorithms at all there must be one for HHL.

However, the exponential speedup of HHL comes with some caveats, thoroughly analysed by Aaronson~\cite{Aaronson15}.
Two of these challenges are common to many quantum algorithms:
1) the efficient encoding of classical data into quantum states and
2) the efficient extraction of classical data via quantum measurements.
Indeed, what HHL really takes as input is not a vector ${b}$ but a quantum state $\ket{b} = \sum_{i=1}^n b_i \ket{i}$ called its amplitude encoding.
Either the input vector ${b}$ has enough structure that we can describe it with a simple, explicit formula.
This is the case for example in the calculation of electromagnetic scattering cross-sections~\cite{CladerEtAl13}.
Or we assume that our classical data has been loaded onto a quantum random-access memory (qRAM) that can prepare the state in logarithmic time~\cite{GiovannettiEtAl08}.
Not only is qRAM a daunting challenge from an engineering point of view, in some cases it also requires too much error correction for the state preparation to be efficient \cite{ArunachalamEtAl15}.
Symmetrically, the output of HHL is not the solution vector ${x}$ itself but a quantum state $\ket{x}$ from which we can measure some observable $\bra{x} M \ket{x}$.
If preparing the state $\ket{b}$ requires a number of gates exponential in the number of qubits, or if we need exponentially many measurements of $\ket{x}$ to compute our classical output, then the quantum speedup disappears.

Shor, Grover and HHL all assume \emph{fault-tolerant} quantum computers~\cite{Shor96}.
Indeed, any machine we can build will be subject to noise when performing quantum operations, errors are inevitable: we need an error correcting code that can correct these errors faster than they appear.
This is the content of the \emph{quantum threshold theorem}~\cite{AharonovBen-Or08} which proves the possibility of fault-tolerant quantum computing given physical error rates below a certain threshold.
One noteworthy example of such a quantum error correction scheme is Kitaev's toric code~\cite{Kitaev03} and the general idea of topological quantum computation~\cite{FreedmanEtAl03} which offers the long-term hope for a quantum computer that is fault-tolerant ``by its physical nature''.
However this hope relies on the existence of quasi-particles called Majorana zero-modes, which as of 2021 has yet to be experimentally demonstrated~\cite{Ball21}.

The road to large-scale fault-tolerant quantum computing will most likely be a long one.
So in the meantime, what can we do with the noisy intermediate-scale quantum machines we have available today, in the so-called NISQ era~\cite{Preskill18}?
Most answers involve a hybrid classical-quantum approach where a classical algorithm is used to optimise the preparation of quantum states~\cite{McCleanEtAl16}.
Prominent examples include the quantum approximate optimisation algorithm (QAOA~\cite{FarhiEtAl14}) for combinatorial problems such as maximum cut and the variational quantum eigensolver (VQE~\cite{PeruzzoEtAl14}) for approximating the ground state of chemical systems.
These variational algorithms depend on the choice of a parameterised quantum circuit called the \emph{ansatz}, based on the structure of the problem and the resources available.
Some families of ansätze such as instantaneous quantum polynomial-time (IQP) circuits are believed to be hard to simulate classically even at constant depth \cite{ShepherdBremner09}, opening the door to potentially near-term NISQ speedups.

Although the hybrid approach first appeared in the context of machine learning~\cite{BangEtAl08}, the idea of using parameterised quantum circuits as machine learning models went mostly unnoticed for a decade~\cite{BenedettiEtAl19}.
It was rediscovered under the name of quantum neural networks~\cite{FarhiNeven18} then implemented on two-qubits~\cite{HavlicekEtAl19}, generating a new wave of attention for quantum machine learning.
The idea is straightforward: 1) encode the input vector ${x} \in \mathbb{R}^n$ as a quantum state $\ket{\phi_{x}}$ via the ansatz of our choice, 2) initialise a random vector of parameters ${\theta} \in \mathbb{R}^d$ and encode it as a measurement $M_\theta$, again via some choice of ansatz 3) take the probability $y = \bra{\phi({x})} M_\theta \ket{\phi({x})}$ as the prediction of the model.
A classical algorithm then uses this quantum prediction as a subroutine to find the optimal parameters $\theta$ in some data-driven task such as classification.

One of the many challenges on the way to solving real-world problems with parameterised quantum circuits is the existence of \emph{barren plateaus}~\cite{McCleanEtAl18}:
with random circuits as ansatz, the probability of non-zero gradients is exponentially small in the number of qubits and our classical optimisation gets lost in a flat landscape.
One can help but notice the striking similarity with the vanishing gradient
problem for classical neural networks, formulated twenty years earlier~\cite{Hochreiter98}.
Barren plateaus do not appear in circuits with enough structure such as quantum convolutional networks~\cite{PesahEtAl21}, they can also be mitigated by structured initialisation strategies~\cite{GrantEtAl19}.
Another direction is to avoid gradients altogether and use \emph{kernel methods}~\cite{SchuldKilloran19}:
instead of learning a measurement $M_\theta$, we use our NISQ device to estimate the distance $|\braket{\phi_{x'}}{\phi_x}|^2$ between pairs of input vectors $x, x' \in \mathbb{R}^n$ embedded in the high-dimensional Hilbert space of our ansatz.
We then use a classical support vector machine to find the optimal hyperplane that separates our data, with theoretical guarantees to learn quantum models at least as good as the variational approach~\cite{Schuld21}.

Random quantum circuits may be unsuitable for machine learning, but they play a crucial role in the quest for \emph{quantum advantage}, the experimental demonstration of a quantum computer solving a task that cannot be solved by classical means in any reasonable time.
We are back to Feynman's original intuition: sampling from a random quantum circuit is the perfect candidate for such a task.
The end of 2019 saw the first claim of such an advantage with a 53-qubit computer~\cite{AruteEtAl19}.
The claim was almost immediately contested by a classical simulation of 54 qubits in two and a half days~\cite{PednaultEtAl19} then in five minutes~\cite{YongEtAl21}.
Zhong et al.~\cite{ZhongEtAl20} made a new claim with a 76-photon linear optical quantum computer followed by another with a 66-qubit computer~\cite{WuEtAl21,ZhuEtAl21}.
They estimate that a classical simulation of the sampling task they completed in a couple of hours would take at least ten thousand years.

Now that quantum computers are being demonstrated to compute something beyond classical, the question remains: can they compute something \emph{useful}?

\input{txt/0-intro/b-make-nlp-quantum}
%!TEX root = ../../THESIS.tex

\section*{How can category theory help?}
\addcontentsline{toc}{section}{How can category theory help?}

\justepigraph{
I should still hope to create a kind of \emph{universal symbolistic} (\emph{spécieuse générale}) in which all truths of reason would be reduced to a kind of calculus.
}{\emph{Letter to Nicolas Remond}, Leibniz (1714)}

``Every sufficiently good analogy is yearning to become a functor''~\cite{Baez06} and we will see that the analogy behind DisCoCat models is indeed a functor.
Coecke et al.~\cite{CoeckeEtAl13} make a meta-analogy between their models of natural language and \emph{topological quantum field theories} (TQFTs).
Intuitively, there is an analogy between regions of spacetime and quantum processes: both can be composed either in sequence or in parallel.
TQFTs formalise this analogy: they assign a quantum system to each region of space and a quantum process to each region of spacetime, in a way that respects sequential and parallel composition.
In the same structure-preserving way, DisCoCat models assign a vector space to each grammatical type and a linear map to each grammatical derivation.
Both TQFTs and DisCoCat can be given a one-sentence definition in terms of category theory: they are examples of \emph{functors into the category of vector spaces}.

How can the same piece of general abstract nonsense (category theory's nickname) apply to both quantum gravity and natural language processing?
And how can this nonsense be of any help in the implementation of QNLP algorithms?
This section will answer with a brief and biased history of category theory and its applications to quantum physics and computational linguistics, from an abstract framework for meta-mathematics to a concrete toolbox for NLP on quantum hardware.
First, a short philosophical digression on the etymology of the words ``functor'' and ``category'' shall bring some light to their divergent meanings in mathematics and linguistics.

The word ``functor'' first appears in Carnap's \emph{Logical syntax of language}~\cite{Carnap37} to describe what would be called a \emph{function symbol} in a modern textbook on first-order logic.
He introduces them as a way to reduce the laws of empirical sciences like physics to the pure syntax of his formal logic, taking the example of a \emph{temperature functor} $T$ such that $T(3) = 5$ means ``the temperature at position 3 is 5''\footnote
{MacLane~\cite{MacLane38} would later remark that Carnap's formal language cannot express the coordinate system for positions, nor the scale in which temperature is measured.}.
In the linguistics community, this meaning has then drifted to become synonymous with \emph{function words} such as ``such'', ``as'', ``with'', etc. These words do not refer to anything in the world but serve as the grammatical glue between the \emph{lexical words} that describe things and actions.
They represent less than one thousandth of our vocabulary but nearly half of the words we speak~\cite{ChungPennebaker07}.

Categories (from the ancient Greek \emph{\foreignlanguage{greek}{κατηγορία}}, ``that which can be said'') have a much older philosophical tradition.
In his \emph{Categories}~\cite{Aristotle66}, Aristotle first makes the distinction between the simple forms of speech (the things that are ``said without any combination'' such as ``man'' or ``arguing'') and the composite ones such as ``a man argued''.
He then classifies the simple, atomic things into ten categories: ``each signifies either substance or quantity or qualification or a relative or where or when or being-in-a-position or having or doing or being-affected''.
A common explanation~\cite{Ryle37} for how Aristotle arrived at such a list is that it comes from the possible \emph{types of questions}: the answer to ``What is it?'' has to be a substance, the answer to ``How much?'' a quantity, etc.
Although he was using language as a tool, his system of categories aims at classifying things in the world, not forms of speech: it was meant as an \emph{ontology}, not a grammar.
In his \emph{Critique of Pure Reason}~\cite{Kant81}, Kant revisits Aristotle's system to classify not the world, but the mind: he defines categories of understanding rather than categories of being.
The idea that every object (whether in the world or in the mind) is an object of a certain type has then become foundational in mathematical logic and Russell's \emph{theory of types}~\cite{Russell03}.
The same idea has also had a great influence in linguistics and especially in the \emph{categorial grammar} tradition initiated by Ajdukiewicz~\cite{Ajdukiewicz35} and Bar-Hillel~\cite{Bar-Hillel53,Bar-Hillel54}, where categories have now become synonymous with \emph{grammatical types} such as nouns, verbs, etc.

Independently of their use in linguistics, Eilenberg and MacLane~\cite{EilenbergMacLane42,EilenbergMacLane42a,EilenbergMacLane45} gave categories and functors their current mathematical definition.
Inspired by Aristotle's categories of things and Kant's categories of thoughts, they defined categories as types of \emph{mathematical structures}: sets, groups, spaces, etc.
Their great insight was to focus not on the content of the objects (elements, points, etc.) but on the composition of the \emph{arrows} between them: functions, homomorphisms, continuous maps, etc.
Applying the same insight to categories themselves, what really matters are the arrows between them: \emph{functors}, maps from one category to another that preserve the form of arrows.\footnote
{We can play the same game again: what matters are not so much the functors themselves but the \emph{natural transformations} between them, which is what category theory was originally meant to define.
To keep playing that game is to fall in the rabbit hole of infinity category theory~\cite{RiehlVerity16}.}
A prototypical example is Poincaré's construction of the fundamental group of a topological space~\cite{Poincare95}, which can be defined as a functor from the category of (pointed) topological spaces to that of groups: every continuous map between spaces induces a homomorphism between their fundamental groups, in a way that respects composition and identity.
Thus, the abstraction of category theory allowed to formalise the analogies between topology and algebra, proving results about one using methods from the other.
It was then used as a tool for the foundation of algebraic geometry by the school of Grothendieck~\cite{GrothendieckDieudonne60}, which brought the analogy between geometric shapes and algebraic equations to a new level of abstraction and led to the development of \emph{topos theory}.

The establishment of category theory as an independent discipline and as a foundation for mathematics owes much to the work of Lawvere.
His influential Ph.D. thesis~\cite{Lawvere63} on \emph{functorial semantics} set up a framework for model theory where logical theories are categories and their models are functors.
He then undertook the axiomatisation of the category of sets~\cite{Lawvere64} and the category of categories~\cite{Lawvere66}.
The resulting notion of elementary topos~\cite{Lawvere70a} subsumed Grothendieck's definition and emphasised the foundational concept of \emph{adjunction}~\cite{Lawvere69a,Lawvere70}.
``Adjoint functors arise everywhere'' became the slogan of MacLane's classic textbook \emph{Categories for the working mathematician}~\cite{MacLane71}.
Lambek~\cite{Lambek68,Lambek69,Lambek72} used the related notion of \emph{cartesian closed categories} to extend the Curry-Howard correspondance between logic and computation into a trinity with category theory: proofs and programs are arrows, logical formulae and data types are objects.
The discovery of this three-fold connection resulted in a wide range of applications of category theory to theoretical computer science, surveyed in Scott~\cite{Scott00}.

This unification of mathematics, logic and computer science has been followed by an ongoing program of categorical foundations for physics, initiated by Lawvere's topos-theoretic treatment of classical dynamics~\cite{Lawvere79} and continuum physics~\cite{LawvereSchanuel86} with Schanuel.
As we mentioned at the start of this section, the work of Atiyah~\cite{Atiyah88}, Baez and Dolan~\cite{BaezDolan95} on TQFTs showed categories and functors to be essential tools in the grand unification project of quantum gravity~\cite{Baez06}.
This now quaternary analogy between physics, mathematics, logic and computation was popularised by Baez and Stay in their \emph{Rosetta Stone}~\cite{BaezStay09}.
On more concrete grounds, this connection between category theory and quantum physics appeared in Selinger's proposal of a quantum programming language~\cite{Selinger04} and the development of a quantum lambda calculus~\cite{VanTonder04,SelingerValiron06,SelingerEtAl09}.
The same insight blossomed in the school of \emph{categorical quantum mechanics} (CQM) led by Abramsky and Coecke~\cite{AbramskyCoecke04,AbramskyCoecke08}, where quantum processes are arrows in \emph{compact closed categories}.
This approach culminated in the \emph{ZX calculus} of Coecke and Duncan~\cite{CoeckeDuncan08,CoeckeDuncan11}, a categorical axiomatisation which was proved complete for qubit quantum computing~\cite{JeandelEtAl18a,HadzihasanovicEtAl18}
with applications including error correction \cite{ChancellorEtAl18,GidneyFowler19}, circuit optimisation~\cite{KissingerVanDeWetering20,DuncanEtAl20,DeBeaudrapEtAl20}, compilation \cite{CowtanEtAl20,DeGriendDuncan20} and extraction \cite{BackensEtAl20}.

In quantum computing as well, adjunction is fundamental: it underlies the definition of entanglement and the proof of correctness for the \emph{teleportation protocol}.
Back in 2004 when Coecke first presented this result at the McGill category theory seminar, Lambek immediately pointed out the analogy with his \emph{pregroup grammars}~\cite{Lambek99,Lambek01} where adjunction is the only grammatical rule\footnote
{See \cite{Coecke19} for a first-hand account of this story and a praise of Jim Lambek.}.
Half a century beforehand, Lambek~\cite{Lambek58,Lambek59,Lambek61} had started to unravel the analogy between the derivations in categorial grammars and proof trees in mathematical logic.
He then extended this analogy in \emph{Categorial and categorical grammar}~\cite{Lambek88} where he showed that these grammatical derivations are in fact arrows in \emph{closed monoidal categories} and proposed to cast Montague semantics as a topos-valued functor.
Later, he argued not ``that categories should play a role in linguistics, but rather that they already do''~\cite{Lambek99b}.
Indeed, Hotz~\cite{Hotz66} had already proved that Chomsky's generative grammars were \emph{free monoidal categories}, although his original German article was never translated to English and remains confidential.
The idea of using functors as semantics had appeared implicitly in Knuth~\cite{Knuth68a} in the context-free case and was made explicit by Benson~\cite{Benson70a} for unrestricted grammars.
From this categorical formulation of linguistics, Lambek~\cite{Lambek10} first suggested the analogy between linguistics and physics which is the basis of this thesis: \emph{pregroup reductions as quantum processes}.

It is remarkable that Lambek could foresee QNLP without \emph{string diagrams}\footnote
{String diagrams do not appear in any of Lambek's published work.
Instead, he either uses lines of equations, proof trees or ``underlinks'' for pregroup adjunctions~\cite{Lambek08}.
He admits ``not having had the patience to absorb'' the topological definition of Joyal-Street string diagrams~\cite{Lambek10}.
}, probably the most powerful tool in the hands of the applied category theorist.
They first appeared in another confidential article from Hotz~\cite{Hotz65} as a formalisation of the diagrams commonly used in electronics.
Penrose~\cite{Penrose71} then used the same notation as an informal shortcut for tedious tensor calculations, and later applied it to relativity theory with Rindler~\cite{PenroseRindler84}.
Joyal and Street~\cite{JoyalStreet88,JoyalStreet91,JoyalStreet95} gave the first topological definition of string diagrams and characterised them as the arrows of free monoidal categories.
At first a piece of mathematical folklore that was hand-drawn on blackboards and rarely included in publications, string diagrams were published at a much bigger scale with the advent of typesetting tools like \LaTeX and Ti\emph{k}Z.
Selinger's survey~\cite{Selinger10}, makes the hierarchy of categorical structures (symmetric, compact closed, etc.) correspond to a hierarchy of graphical gadgets (swaps, wire bending, etc.).
In \emph{Picturing Quantum Processes}~\cite{CoeckeKissinger17}, Coecke and Kissinger introduce quantum theory with over a thousand diagrams.
And the list of applications keeps growing:
electronics~\cite{BaezFong15} and chemistry~\cite{BaezPollard17},
control theory~\cite{BaezErbele14} and concurrency~\cite{BonchiEtAl14a},
databases~\cite{BonchiEtAl18} and knowledge representation \cite{Patterson17},
Bayesian inference~\cite{CoeckeSpekkens12,ChoJacobs19} and causality~\cite{KissingerUijlen19},
cognition~\cite{BoltEtAl17} and game theory~\cite{GhaniEtAl18},
functional programming \cite{Riley18} and machine learning~\cite{FongEtAl17}.

If they are a great tool for writing scientific papers, string diagrams can also be a powerful data structure for developing software applications:
quantomatic~\cite{KissingerZamdzhiev15} and its successor PyZX~\cite{KissingerVanDeWetering19} perform automatic rewriting of diagrams in the ZX calculus,
globular~\cite{BarEtAl18} and its successor homotopy.io~\cite{ReutterVicary19} are proof assistants for higher category theory,
cartographer~\cite{SobocinskiEtAl19} and catlab~\cite{PattersonEtAl21} implement diagrams in symmetric monoidal categories, which are also implicit in the circuit data structure of the t$|$ket$\rangle$ compiler~\cite{SivarajahEtAl20}.
String diagrams are the main data structure of our QNLP algorithms: we translate the diagrams of sentences into diagrams of quantum circuits.
As none of the existing category theory software was flexible enough, we had to implement our own: DisCoPy~\cite{FeliceEtAl20}, a Python library for computing with functors and diagrams in monoidal categories.
DisCoPy then became the engine underlying lambeq~\cite{KartsaklisEtAl21}, a high-level library for experimental QNLP.
Although its development was driven by the implementation of DisCoCat models on quantum computers, DisCoPy was designed as a general-purpose toolkit for applied category theory.
It is freely available\footnote{\url{https://github.com/oxford-quantum-group/discopy}} (as in free beer and in free speech), reliable (with 100\% code coverage) and extensively documented\footnote{\url{https://discopy.readthedocs.io/}}.

In conclusion, category theory can really be a \emph{theory of anything}: from algebraic geometry and quantum gravity to natural language processing.
There is a striking analogy between category theory and string diagrams as a universal graphical language and the \emph{characteristica universalis} and \emph{calculus ratiocinator} dreamt by Leibniz three hundred years ago, a formal language and computational framework that would be able to express all of mathematics, science and philosophy.
Indeed, not only can categories be tools for the working mathematicians and scientists, they can also be of help to the philosophers.
In the footsteps of Grassmann's \emph{Ausdehnungslehre}~\cite{Grassmann44} and his project of an algebraic formalisation of Hegel, Lawvere~\cite{Lawvere89,Lawvere91,Lawvere92,Lawvere96} set out to formulate Hegelian dialectics in terms of adjunctions.
This led to the ongoing effort of Schreiber, Corfield and their collaborators on the nLab~\cite{SchreiberEtAl21} to translate \emph{Wissenschaft Der Logik}~\cite{Hegel12} in terms of category theory.
Not only can it accommodate the absolute idealism of Hegel, category theory can also deal with the pragmatism of Peirce~\cite{Peirce06},
who developed first-order logic independently of Frege using what was later recognised as the first string diagrams~\cite{BradyTrimble98,BradyTrimble00,MelliesZeilberger16,HaydonSobocinski20}.
String diagrams have also been used to model Wittgenstein's language games as functors from a grammar to a category of games~\cite{HedgesLewis18}.
In recent work~\cite{FeliceEtAl20a}, we applied these functorial language games to question answering, going from philosophy to NLP via category theory.

%!TEX root = ../../THESIS.tex

\section*{Contributions}
\addcontentsline{toc}{section}{Contributions}

The first chapter is an extended version of the DisCoPy paper~\cite{FeliceEtAl20a}.
It emerged from a dialectic teacher-student collaboration with Giovani de Felice: implementing our own category theory library was a way to teach him Python programming.
Bob Coecke then added the capital letters to the name of DisCoPy.
\begin
{itemize}
\item We\footnote
{The ``we'' of this section refers to the author of this thesis.
Although we believe that science is collaboration and that the notion of personal contribution is obsolete, it is in fact required by university regulations: ``Where some part of the thesis is not solely the work of the candidate or has been carried out in collaboration with one or more persons, the candidate shall submit a clear statement of the extent of his or her own contribution.''} give an elementary definition of string diagrams for monoidal categories.
Our construction decomposes the free monoidal category construction into three basic steps: 1) a layer monad on the category of monoidal signatures, 2) the free premonoidal category as a free category of layers and 3) the free monoidal category as a quotient by interchangers.
To the best of our knowledge, this \emph{premonoidal approach} had been relegated to mathematical folklore: it was known by those who knew, yet it never appeared in print.
\item We prove the equivalence between our elementary definition and the topological definition of Joyal and Street~\cite{JoyalStreet88}.
One side of this equivalence underlies the drawing algorithm of DisCoPy, the other side is the basis of a prototype for an automatic diagram recognition algorithm.
\item We describe our object-oriented implementation of monoidal category theory.
The hierarchy of categorical structures (monoidal, closed, rigid, etc.) is encoded in a hierarchy of Python classes and an inheritance mechanism implements the free-forgetful adjunctions between them.
\item We discuss the relationship between our premonoidal approach and the existing graph-based data structures for diagrams in symmetric monoidal categories.
\end
{itemize}
The second chapter deals with QNLP, building on \cite{MeichanetzidisEtAl20,CoeckeEtAl20,MeichanetzidisEtAl20a}.
It was joint work with Bob Coecke, Giovanni de Felice and Konstantinos Meichanetzidis.
Although we were working in the same office, Stefano Gogioso arrived at the same ideas independently with his collaborator Nicolò Chiappori.
\begin
{itemize}
\item We define QNLP models as functors from grammar to quantum circuits and show that any DisCoCat model can be implemented in this way.
\item We develop a rewriting strategy for the resulting circuits which reduces both the required number of qubits and the amount of post-selection.
The underlying algorithm, called \emph{snake removal}, computes the normal form of diagrams in rigid monoidal categories.
\item We introduce a hybrid classical-quantum algorithm to train QNLP models on a question-answering task.
The underlying idea of \emph{functorial learning}, i.e. learning structure-preserving functors from diagram-like data, provides a theoretical framework for machine learning on structured data.
\end
{itemize}
The third chapter introduces \emph{diagrammatic differentiation}, a graphical calculus for computing the gradients of parameterised diagrams which applies to the training of QNLP models but also to functorial learning in general.
Most of the material has been published in joint work with Richie Yeung and Giovanni de Felice~\cite{ToumiEtAl21a}.
\begin
{itemize}
\item We generalise the dual number construction from rings to monoidal categories. Dual diagrams are formal sums of a string diagram (the real part) and its derivative with respect to some parameter (the epsilon part).
\item We introduce graphical gadgets called bubbles, which can encode arbitrary unary operators on monoidal categories.
In particular, they encode differentiation of diagrams and allow to express the standard rules of calculus (linearity, product, chain) entirely in terms of diagrams.
\item We study diagrammatic differentiation for the ZX calculus.
In the pure case, this allows to compute the gradients of linear maps with respect to phase parameters.
In the mixed classical-quantum case, this yields a definition of the parameter-shift rules used in quantum machine learning.
\item We define the gradient of QNLP models and parameterised functors in general.
\end{itemize}

\section*{Publications}
\addcontentsline{toc}{section}{Publications}

The material presented in this thesis builds on the following publications.
\begin{itemize}[label={}]
\item \enumcite{MeichanetzidisEtAl20a}\vspace{-10pt}
\item \enumcite{FeliceEtAl20a}\vspace{-10pt}
\item \enumcite{CoeckeEtAl20}\vspace{-10pt}
\item \enumcite{MeichanetzidisEtAl20}\vspace{-10pt}
\item \enumcite{ToumiEtAl21a}
\end{itemize}\vspace{10pt}
During his DPhil, the author has also published the following articles.
\begin{itemize}[label={}]
\item \enumcite{FeliceEtAl19}\vspace{-10pt}
\item \enumcite{FeliceEtAl20}\vspace{-10pt}
\item \enumcite{ShieblerEtAl20}
\item \enumcite{KartsaklisEtAl21}\vspace{-10pt}
\item \enumcite{ToumiKoziell-Pipe21}
\item \enumcite{CoeckeEtAl21}\vspace{-10pt}
\item \enumcite{McPheatEtAl21}
\end{itemize}


\chapter{DisCoPy: Python for the applied category theorist}

%!TEX root = ../../THESIS.tex

\chapter{DisCoPy: Python for the applied category theorist}\label{chapter:discopy}

Python has become the programming language of choice for most applications in both natural language processing (e.g. Stanford NLP~\cite{ManningEtAl14}, NLTK~\cite{LoperBird02} and SpaCy~\cite{HonnibalMontani17}) and quantum computing (with development kits like Qiskit~\cite{Cross18} and PennyLane~\cite{BergholmEtAl20} and interfaces to compilers like pytket~\cite{SivarajahEtAl20}).
Thus, it was the obvious choice of language for an implementation of QNLP.
However, unlike functional programming languages like Haskell, Python has little support for category theory.
Indeed, before the release of DisCoPy, the only existing Python framework for category theory was a module of SymPy~\cite{MeurerEtAl17} that can draw commutative diagrams in finite categories.
Hence, the first step in implementing QNLP was to develop our own framework for applied category theory in Python: DisCoPy.
Its main feature are the drawing of string diagrams (e.g. the grammatical structure of sentences) and the application of functors (e.g. to quantum circuits, either executed on quantum hardware or classically simulated).

String diagrams have become the lingua franca of applied category theory.
However, the definitions one can find in the literature usually fall into one of two extremes: either definitions by general abstract nonsense or definitions by example and appeal to intuition.
On one side of the spectrum, the standard technical reference has become the \emph{Geometry of tensor calculus}~\cite{JoyalStreet91} where Joyal and Street define string diagrams as equivalence classes of labeled topological graphs embedded in the plane and then characterise them as the arrows of free monoidal categories.
On the other, \emph{Picturing quantum processes}~\cite{CoeckeKissinger17} contains over a thousand string diagrams but their formal definition as well as any mention of category theory are relegated to mere appendices.

The aims of this chapter are three-fold: 1) it gives an overview of the DisCoPy package and its design principles, 2) it introduces elementary category theory to the Python programmer and 3) it introduces object-oriented programming to the applied category theorist.
The first section introduces categories and functors with no mathematical prerequisites apart from sets and monoids.
The second section introduces monoidal categories, defining string diagrams from first principles.
The third section defines the drawing and reading algorithms for string diagrams, which arise as the two sides of the equivalence between the premonoidal and the topological definitions.
The fourth section introduces monoidal categories with extra structure and the inheritence mechanism which implements this hierarchy of structure.
The fifth section gives the category theoretic foundations for our definition of diagrams, which we call the premonoidal approach, it discusses the relationship between this approach and the exisiting graph-based data structures for diagrams in symmetric monoidal categories.

Note that the code presented in this thesis represents a significant refactoring of the original implementation of DisCoPy as available online at the time this thesis is submitted\footnote
{\url{https://github.com/oxford-quantum-group/discopy}}.
We list some of the significant changes between the two versions:
\begin{itemize}
\item better inheritance with staticmethods
\item less bureaucracy with dataclass
\item more uniform syntax inside, dom, cod
\item matrices and functions no longer subclasses of Box, Composable and Tensorable for syntactic sugar
\item 
\end{itemize}

%!TEX root = ../../THESIS.tex

\section{Categories in Python}

What are categories and how can they be useful to the Python programmer?
This section will answer this question by taking the standard mathematical definitions and breaking them into \emph{data}, which can be translated into Python code, and \emph{axioms}, which cannot be formally verified in Python, but can be translated into test cases.
The data for a category is given by a tuple $C = (C_0, C_1, \dom, \cod, \id, \then)$ where:
\begin{itemize}
\item $C_0$ and $C_1$ are classes of \emph{objects} and \emph{arrows} respectively,
\item $\dom, \cod : C_1 \to C_0$ are functions called \emph{domain} and \emph{codomain},
\item $\id : C_0 \to C_1$ is a function called \emph{identity},
\item $\then : C_1 \times C_1 \to C_1$ is a partial function called \emph{composition}, denoted by $(\fcmp)$.
\end{itemize}
Given two objects $x, y \in C_0$, the set\footnote
{We will assume that this forms a set rather than a proper class, i.e. we will only work with \emph{locally small} categories.}
$C(x, y) = \{f \in C_1 \s \vert \s \dom(f), \cod(f) = x, y \}$ is called a \emph{homset} and we write $f : x \to y$ whenever $f \in C(x, y)$.
We denote the composition $\then(f, g)$ by $f \fcmp g$, translated to \py{f >> g}  in Python.
The axioms for the category $C$ are the following:
\begin{itemize}
\item $\id(x) : x \to x$ for all objects $x \in C_0$,
\item for all arrows $f, g \in C_1$, the composition $f \fcmp g$ is defined iff $\cod(f) = \dom(g)$, moreover we have $f \fcmp g : \dom(f) \to \cod(g)$,
\item $\id(\dom(f)) \fcmp f = f = f \fcmp \id(\cod(f))$ for all arrows $f \in C_1$,
\item $f \fcmp (g \fcmp h) = (f \fcmp g) \fcmp h$ whenever either side is defined for $f, g, h \in C_1$.
\end{itemize}

Note that we play with the overloaded meaning of the word \emph{class}: we use it to mean both a mathematical collection that need not be a set, and a Python class with its methods and attributes.
Reading it in the latter sense, $\dom$ and $\cod$ are \emph{attributes} of the arrow class, $\then$ is a \emph{method}, $\id$ is a \emph{static method}.
Thus, implementing a category in Python means nothing more than subclassing the  abstract classes \py{Object} and \py{Arrow} of listing~\ref{listing:abstract classes}, and then checking that the axioms hold via some (necessarily non-exhaustive) software tests.

\begin{python}\label{listing:abstract classes}
{\normalfont Abstract classes for categories, functors and transformations.}

Note that annotations with dependent types are not supported by any Python implementation yet.
Since Python could not statically check that compositions are well-typed, DisCoPy has no type hints and raises an \py{AxiomError} at runtime instead.
\begin{minted}{python}
class Object: ...

class Arrow:
    dom: Object, cod: Object

    @staticmethod
    def id(x: Object) -> Arrow[x, x]: ...

    def then(self, other: Arrow[self.cod, y]) -> Arrow[self.dom, y]: ...

class Functor:
    @overload
    def __call__(self, x: Object) -> Object: ...

    @overload
    def __call__(self, f: Arrow[x, y]) -> Arrow[self(x), self(y)]: ...

class Transformation:
    dom: Functor, cod: Functor

    def __call__(self, x: Object) -> Arrow[self.dom(x), self.cod(x)]: ...
\end{minted}
\end{python}

The data for a \emph{functor} $F : C \to D$ between two categories $C$ and $D$ is given by a pair of overloaded functions $F : C_0 \to D_0$ and $F : C_1 \to D_1$ such that:
\begin{itemize}
    \item $F(\dom(f)) = \dom(F(f))$ and $F(\cod(f)) = \cod(F(f))$ for all $f \in C_1$,
    \item $F(\id(x)) = \id(F(x))$ and $F(f \fcmp g) = F(f) \fcmp F(g)$ for all $x \in C_0$ and $f, g \in C_1$.
\end{itemize}
Thus, implementing a functor in Python amounts to subclassing the \py{Functor} class of listing~\ref{listing:abstract classes} (and then implementing software tests to check that the axioms hold).

The data for a \emph{transformation} $\alpha : F \to G$ between two parallel functors $F, G : C \to D$ is given by a function from objects $x \in C_0$ to components $\alpha(x) : F(x) \to G(x)$ in $D$.
A \emph{natural transformation} is one where $\alpha(x) \fcmp G(f) = F(f) \fcmp \alpha(y)$ for all arrows $f : x \to y$ in $C$.
The \py{Transformation} class is given in listing~\ref{listing:abstract classes}, checking that a transformation is natural cannot be done formally in Python.
In the same way that there is a set $Y^X$ of functions $X \to Y$ for any two sets $X$ and $Y$, for any two categories $C$ and $D$ there is a category $D^C$ with functors $C \to D$ as objects and natural transformations as arrows.

\begin{example}\label{ex:python categories}
We can define the category $\mathbf{Pyth}$ with objects the class of all Python types and arrows the class of all Python functions.
Domain and codomain of may be extracted from type annotations.
Identity may given by \py{lambda *xs: xs} and the composition by \py{lambda f, g: lambda *xs: f(*g(*xs)))}. (The star takes care of functions with multiple arguments.)
However, equality of functions in Python is undecidable so there will be no way to check the axioms hold in general.

Endofunctors $\mathbf{Pyth} \to \mathbf{Pyth}$ can be thought of as some kind of data containers.
For example, we can define a $\mathbf{List}$ functor which sends a type \py{t} to \py{List[t]} and a function \py{f} to \py{lambda *xs: map(f, xs)}.

There is a natural transformation $\eta : \mathbf{Id} \to \mathbf{List}$ from the obvious identity functor, implemented by the built-in function \py{id}.
Its components send objects \py{x : t} of any type \py{t} to the singleton list \py{[x] : List[t]}.
\end{example}

\begin{python}
{\normalfont Implementation of the category $\mathbf{Pyth}$ with \py{type} as objects and \py{Function} as arrows.}

\begin{minted}{python}
@dataclass
class Function:
    dom: type
    cod: type
    inside: Callable

    @staticmethod
    def id(x: type) -> Function:
        return Function(x, x, lambda *xs: xs)

    def then(self, other: Function) -> Function:
        return Function(self.dom, other.cod, lambda *xs: other(*self(*xs)))

    def __call__(self, *xs): return self.inside(*xs)

f = Function(int, Iterable, range)
g = Function(Iterable, int, sum)
h = Function(int, int, lambda n: n * (n - 1) // 2)
assert f.then(g)(42) == h(42) == 861
\end{minted}
\end{python}

\begin{example}
When the class of objects and arrows are in fact sets, $C$ is called a \emph{small category}.
For example, the category $\mathbf{FinSet}$ has the set of all finite sets as objects and the set of all functions between them as arrows.
This time equality of functions between finite sets is decidable, so we can write unit tests that check that the axioms hold on specific examples.
\end{example}

\begin{example}
When the class of objects and arrow are finite sets, we can draw the category as a directed multigraphs with objects as vertices and arrows as edges, together with the list of equations between paths.
A functor $F : C \to D$ from such a finite category $C$ is called a \emph{commutative diagram} in $D$.
For example, the following commutative diagram denotes a functor $3 \to \mathbf{Pyth}$ from the finite category $3$ with three objects $\{ 0, 1, 2 \}$ and three non-identity arrow $f : 0 \to 1, g : 1 \to 2$ and $h : 0 \to 2$, with the only non-trivial composition $f \fcmp g = h$.
\[ \begin{tikzcd}
\py{int}
\ar{rrrrrr}{\py{lambda n: n * (n - 1) // 2}}
\ar{rrrd}[']{\py{range}}
& & & & & & \py{int} \\
& & & \py{Iterable}
\ar{urrr}[']{\py{sum}} & & &
\end{tikzcd}
\]
Thus, this commutative diagram is the equation \py{sum(range(n)) = n * (n - 1) // 2}.
When the finite category is bigger than a triangle, one commutative diagram can state a large number of equations, which can be read by \emph{diagram chasing}.
\end{example}

\begin{example}
The category $\mathbf{Mat}_\S$ has natural numbers as objects and $n \times m$ matrices with values in $\S$ as arrows $n \to m$.
The identity and composition are given by the identity matrix and matrix multiplication respectively.
In order for matrix multiplication to be well-defined and for $\mathbf{Mat}_\S$ to be a category, the scalars $\S$ should have at least the structure of a \emph{rig} (a riNg without Negatives): a pair of monoids $(\S, +, 0)$ and $(\S, \times, 1)$ with the first one commutative and the second a homomorphism for the first, i.e. $a \times 0 = 0 = 0 \times a$ and $(a + b) \times (c + d) = a c + a d + b c + b d$.
The category $\mathbf{Mat}_\C$ is equivalent to the category of finite dimensional vector spaces and linear maps.
When the scalars are Booleans with disjunction and conjunction as addition and multiplication, the category $\mathbf{Mat}_\B$ is equivalent to the category of finite sets and relations.
There is a faithful functor (i.e. injective on arrows with the same domain and codomain) $\mathbf{FinSet} \to \mathbf{Mat}_\B$ which sends finite sets to their cardinality and functions to their graph.
\end{example}

\begin{python}
{\normalfont Implementation of $\mathbf{Mat}_\B$ with \py{int} as objects and \py{Matrix} as arrows.}

\begin{minted}{python}
@dataclass
class Matrix:
    dom: int
    cod: int
    inside: list[list[bool]]

    @staticmethod
    def id(x: int) -> Matrix:
        return Matrix(x, x, [
            [i == j for i in range(x)] for j in range(x)])

    def then(self, other: Matrix) -> Matrix:
        return Matrix(self.dom, other.cod, [[
            any(self[i][j] and other[j][k] for j in range(other.dom))
            for k in range(other.cod)] for i in range(self.dom)])

    __getitem__ = lambda self, key: self.inside[key]

f = Matrix(1, 2, [[0, 1]])
g = Matrix(2, 2, [[0, 1], [1, 0]])
h = Matrix(2, 1, [[1], [0]])
assert f.then(g) == h
\end{minted}
\end{python}

\begin{example}
The category $\mathbf{Circ}$ has natural numbers as objects and $n$-qubit quantum circuits as arrows $n \to n$.
There is a functor $\mathtt{eval} : \mathbf{Circ} \to \mathbf{Mat}_\C$ which sends $n$ qubits to $2^n$ dimensions and evaluates each circuit to its unitary matrix.
\end{example}

\begin{example}
Just about any class of mathematical structures as objects and their homomorphisms as arrows will form a category.
For example, the category $\mathbf{Set}$ of sets and functions, the category $\mathbf{Mon}$ of monoids and homomorphisms, the category $\mathbf{Cat}$ of small categories and functors, etc.
The faithful functor $U : \mathbf{Mon} \to \mathbf{Set}$ which sends monoids to their underlying set and homomorphisms to functions is called a \emph{forgetful functor}.
\end{example}

The main principles behind the implementation of DisCoPy follow from the concept of a \emph{free object}.
Let's start from a simple example.
Given a set $X$, we can construct a monoid $X^\star$ with underlying set $\coprod_{n \in \N} X^n$ the set of all finite lists with elements in $X$.
The associative multiplication is given by list concatenation $X^m \times X^n \to X^{m + n}$ and the unit is given by the empty list denoted $1 \in X^0$.
Given a function $f : X \to Y$, we can construct a homomorphism $f^\star : X^\star \to Y^\star$ defined by element-wise application of $f$ (this is what the built-in \py{map} does in Python).
We can easily check that $(f \fcmp g)^\star = f^\star \fcmp g^\star$ and $(\id_X)^\star = \id_{X^\star}$.
Thus, we have defined a functor $F : \mathbf{Set} \to \mathbf{Mon}$.

Why is this functor so special? Because it is the \emph{left adjoint} to the forgetful functor $U : \mathbf{Mon} \to \mathbf{Set}$.
An \emph{adjunction} $F \dashv U$ between two functors $F : C \to D$ and $U : D \to C$ is a pair of natural transformations $\eta : \id_C \to F \fcmp U$ and $\epsilon : U \fcmp F \to \id_D$ called the \emph{unit} and \emph{counit} respectively.
In the case of lists, we already mentioned the unit in example~\ref{ex:python categories}: it is the function that sends every object to a singleton list.
For a monoid $M$, the counit $\epsilon(M) : F(U(M)) \to M$ is the monoid homomorphism that takes lists of elements in $M$ and multiplies them.
We can easily check that these two transformations are indeed natural, thus we get that \emph{lists are free monoids}.
This may be taken as a mathematical explanation for why lists are so ubiquitous in programming.
Another equivalent definition of adjunction is in terms of an isomorphism $C(x, U(y)) \simeq D(F(x), y)$ which is natural\footnote
{The isomorphism $C(x, U(y)) \simeq D(F(x), y)$ is natural in $x$ if it is a natural transformation between the two functors $C(-, U(y)), D(F(-), y) : C \to \mathbf{Set}$.}
in $x \in C_0$ and $y \in D_0$.
In the adjunction for lists, functions $X \to U(M)$ from a set $X$ to the underlying set of a monoid $M$ are in a natural one-to-one correspondance with monoid homomorphisms $X^\star \to M$.
To define a homomorphism from a free monoid, it is sufficient to define the image of each generating element.

Now we want to play the same game with categories instead of monoids.
We can define a forgetful functor $U : \mathbf{Cat} \to \mathbf{Set}$ which sends a small category $C$ to its set of objects $C_0$, and its left adjoint $F : \mathbf{Set} \to \mathbf{Cat}$ which sends a set to the \emph{discrete category} with its elements as objects and only identity arrows.
However, this is a rather boring construction because forgetting the arrows of a categories is too much: the forgetful functor $U$ is not faithful.
Instead, we need to replace the category of sets with the category of \emph{signatures}.
The data for a signature is given by a tuple $\Sigma = (\Sigma_0, \Sigma_1, \dom, \cod)$ where:
\begin{itemize}
    \item $\Sigma_0$ is a set of \emph{generating objects},
    \item $\Sigma_1$ is a set of \emph{generating arrows}, which we will also call \emph{boxes},
    \item $\dom, \cod : \Sigma_1 \to \Sigma_0$ are the domain and codomain.
\end{itemize}
A morphism of signatures $f : \Sigma \to \Gamma$ is a pair of overloaded functions $f : \Sigma_0 \to \Gamma_0$ and $f : \Sigma_1 \to \Gamma_1$ such that $f \fcmp \dom = \dom \fcmp f$ and $f \fcmp \cod = \cod \fcmp f$.
Thus, signatures and their morphisms form a category $\mathbf{Sig}$ and there is a faithful functor $U : \mathbf{Cat} \to \mathbf{Sig}$ which sends a category to its underlying signature: it forgets the identity and composition.
Signatures may be thought of as directed multigraphs \emph{with an attitude}~\cite{NLab}.
Given a signature $\Sigma$, we can define a category $F(\Sigma)$ with vertices as objects and \emph{paths as arrows}.
More precisely, an arrow $f : x \to y$ is given by a length $n \in \N$ and a list $f_1, \dots, f_n \in \Sigma_1$ with $\dom(f_1) = x$, $\cod(f_n) = y$ and $\cod(f_i) = \dom(f_{i + 1})$ for all $i < n$.
Given a morphism of signatures $f : \Sigma \to \Gamma$, we get a functor $F(f) : F(\Sigma) \to F(\Gamma)$ relabeling boxes in $\Sigma$ by boxes in $\Gamma$.
Thus, we have defined a functor $F : \mathbf{Sig} \to \mathbf{Cat}$, it remains to show that it indeed forms an adjunction $F \dashv U$.
This is very similar to the monoid case: the unit sends a box in a signature to the path of just itself, the counit sends a path of arrows in a category to their composition.
Equivalently, we have a natural isomorphism $\mathbf{Cat}(F(\Sigma), C) \simeq \mathbf{Sig}(\Sigma, U(C))$: to define a functor $F(\Sigma) \to C$ from a free category is the same as to define a morphism of signatures $\Sigma \to U(C)$.

If lists are such fundamental data structures because they are free monoids, we argue that the arrows of free categories should be just as fundamental: they capture the basic notion of \emph{data pipelines}.
Free categories are implemented in the most basic module of DisCoPy, \py{discopy.cat}, which is sketched in listing~\ref{listing:cat.py}.

\begin{python}~\label{listing:cat.py}
{\normalfont Outline of the classes \py{Ob}, \py{Arrow} and \py{Box}.}
\begin{minted}{python}
@dataclass
class Ob:
    name: str

@dataclass
class Arrow:
    dom: Ob
    cod: Ob
    boxes: list[Arrow]

    @staticmethod
    def upgrade(old: Arrow) -> Arrow:
        return old

    @staticmethod
    def id(x: Ob) -> Arrow:
        return self.upgrade(Arrow(x, x, []))

    def then(self, *others: Arrow) -> Arrow:
        if not others: return self
        for left, right in zip((self, ) + others, others):
            assert left.cod == right.dom
        return self.upgrade(Arrow(
            self.dom, others[-1].cod, self.boxes + sum(
                [other.boxes for other in others], [])))

    __rshift__ = then

class Box(Arrow):
    def __init__(self, name: str, dom: Ob, cod: Ob):
        self.name = name
        super().__init__(dom, cod, [self])

    def __eq__(self, other):
        if not isinstance(other, Arrow): return False
        if isinstance(other, Box):
            return (self.name, self.dom, self.cod)\
                == (other.name, other.dom, other.cod)
        return other.boxes == [self]
\end{minted}
\end{python}

The classes \py{Ob} and \py{Arrow} for objects and arrows are implemented in a straightforward way, using the built-in \py{dataclass} decorator to avoid the bureaucracy of defining initialisation, equality, etc.
The method \py{Arrow.then} accepts any number of arrows \py{others}, which will prove useful when defining functors.
The \py{Box} class requires more attention: a box \py{f = Box('f', x, y)} is an arrow with the list of just itself as boxes, i.e. \py{f.boxes == [f]}.
In order for the axiom \py{f >> Id(y) == f == Id(x) >> f} to hold, we need to make sure that \py{f == Arrow(x, y, [f])}, i.e. a box is set to be equal to the arrow with just itself as boxes.
The main subtlety in the implementation is the method \py{Arrow.upgrade} which for now is just the identity.
When the user defines a subclass of \py{Arrow}, overriding the \py{upgrade} method allows the composition of objects in the subclass to remain within the subclass, without having to rewrite the methods for identity and composition.

\begin{example}
We can define \py{Circuit} as a subclass of \py{Arrow} with only the \py{upgrade} method overriden, and \py{Gate} as a subclass of \py{Circuit} and \py{Box} defined by a name and a number of qubits.
Now we can compose gates together and the result will be an instance of \py{Circuit} not merely of \py{Arrow}.

\begin{minted}{python}
class Circuit(Arrow):
    def upgrade(old):
        return Circuit(old.dom, old.cod, old.boxes)

class Gate(Circuit, Box):
    def __init__(self, name: str, n_qubits: int):
        Box.__init__(self, name, Ob(str(n_qubits)), Ob(str(n_qubits)))
        Circuit.__init__(self, self.dom, self.cod, self.boxes)

H, Z, X = Gate("H", 1), Gate("Z", 1), Gate("X", 1)
assert isinstance(H >> Z >> X, Circuit)
\end{minted}
\end{example}

The \py{Functor} class listed in \ref{listing:Functor} has two mappings \py{ob} and \py{ar} as attributes, from objects to objects and from boxes to arrows respectively.
The domain of the functor is implicitly defined as the free category generated by the domain of the \py{ob} and \py{ar} mappings.
The optional arguments \py{ob_factory} and \py{ar_factory} serve to define functors with arbitrary categories as codomain.
At this point, their only use is for \py{ar_factory} to define identity arrows, otherwise the codomain of the functor is defined implicitly by the codomain of the \py{ob} and \py{ar} mappings.

\begin{python}~\label{listing:Functor}
{\normalfont Outline of the \py{Functor} class.}
\begin{minted}{python}
@dataclass
class Functor:
    ob: dict[Ob, Ob]
    ar: dict[Box, Arrow]
    ob_factory, ar_factory = Ob, Arrow

    def __call__(self, other):
        if isinstance(other, Ob):
            return self.ob[other]
        if isinstance(other, Box):
            return self.ar[other]
        if isinstance(other, Arrow):
            return self.ar_factory.id(self(other.dom)).then(
                *self(box) for box in other.boxes)
        raise TypeError
\end{minted}
\end{python}

\begin{example}
A typical DisCoPy script starts by defining objects and boxes:
\begin{minted}{python}
x, y, z = map(Ob, "xyz")
f, g, h = Box('f', x, y), Box('g', y, z), Box('h', z, x)
\end{minted}
We can define a simple relabeling functor from the free category to itself:
\begin{minted}{python}
F = Functor(
    ob={x: y, y: z, z: x},
    ar={f: g, g: h, h: f})
assert F(f >> g >> h) == F(f) >> F(g) >> F(h) == g >> h >> f
\end{minted}
We can interpret our arrows as Python functions:
\begin{minted}{python}
G = Functor(
    ob={x: int, y: Iterable, z: int},
    ar={f: range, g: sum, h: lambda n: n * (n - 1) // 2},
    ob_factory=type, ar_factory=Function)
assert G(f >> g)(42) == G(h)(42) == 861
\end{minted}
We can interpret our arrows as matrices:
\begin{minted}{python}
H = Functor(
    ob={x: 1, y: 2, z: 2},
    ar={f: [[0, 1]], g: [[0, 1], [1, 0]], h: [[1], [0]]},
    ob_factory=int, ar_factory=Matrix)
assert H(f >> g) == H(h)
\end{minted}
Provided we implement the methods \py{Functor.id} and \py{Functor.then}, we can even build functors into $\mathbf{Cat}$, i.e. interpret arrows as functors:
\begin{minted}{python}
I = Functor(
    ob={x: Arrow, y: Arrow, z: Tensor},
    ar={f: F, g: H}, ar_factory=Functor)
assert I(f >> g)(h) == H(F(h)) == H(f)
\end{minted}
\end{example}

After free objects, another concept behind DisCoPy is that of a \emph{quotient object}.
Again, let's start with the example of a monoid $M$.
Suppose we're given a binary relation $R \sub M \times M$, then we can construct a quotient monoid $M / R$ with underlying set the equivalence classes of the smallest congruence generated by $R$.
That is, the smallest relation $(\sim_R) \sub M \times M$ such that:
\begin{itemize}
\item $x \sim_R y$ for all $(x, y) \in R$,
\item $x \sim_R x$ and if $x \sim_R y$ and $y \sim_R z$ then $x \sim_R z$,
\item if $x \sim_R x'$ and $y \sim_R y'$ then $x \times y \sim_R x' \times y'$.
\end{itemize}
The first point says that $R \sub (\sim_R)$.
The second says that $(\sim_R)$ is an equivalence relation.
The third says that $(\sim_R)$ is closed under products, it is equivalent to the substitution axiom: if $x \sim_R y$ then $a x b \sim_R a y b$ for all $a, b \in M$.
Explicitly, the congruence $(\sim_R)$ can be constructed in two steps: first, we define the rewriting relation $(\to_R) \sub M \times M$ where $a x b \to_R a y b$ for all $(x, y) \in R$ and $a, b \in M$.
Second, we define $(\sim_R)$ as the \emph{symmetric, reflexive, transitive closure} of the rewriting relation, i.e. two elements $x, y \in M$ are equal in $M / R$ iff they are in the same connected component of the undirected graph induced by $(\to_R) \sub M \times M$.
Now there is a homomorphism $q : M \to M / R$ which sends monoid elements to their equivalence class with the following property: for any homomorphism $f : M \to N$ with $x \sim_R y$ implies $f(x) = f(y)$, there is a unique $f' : M / R \to N$ with $f = q \fcmp f'$.
Intuitively, a homomorphism from a quotient $M / R$ is nothing more than a homomorphism from $M$ which respects the axioms $R$.
Up to isomorphism, we can construct any monoid $M$ as the quotient $X^\star / R$ of a free monoid $X^\star$: take $X = U(M)$ and $R = \{ (x y, z) \in X^\star \times X^\star \s \vert \s x \times y = z \in M \}$.

The pair $(X, R \sub X^\star \times X^\star)$ of a set of generating elements $X$ and a binary relation $R$ on its free monoid is called a \emph{presentation} of the monoid $M \simeq X^\star / R$.
Arguably, the most fundamental computational problem is the \emph{word problem for monoids}: given a presentation $(X, R)$ and a pair of lists $x, y \in X^\star$, decide whether $x = y$ in $X^\star / R$.
As mentioned in the introduction, it was shown to be equivalent to Turing's halting problem, and thus undecidable, by Post~\cite{Post47} and Markov~\cite{Markov47}.
The proof is straightforward: we can encode the tape alphabet and the states of a Turing machine in the set $X$ and its transition table into the relation $R$, then whether the machine halts reduces to deciding $x = y$ for $x$ and $y$ the initial and accepting configurations respectively: a proof of equality corresponds precisely to a run of the Turing machine.

The case of quotient categories is similar, only we need to take care of objects now.
Given a category $C$ and a family of binary relations $\{ R_{x,y} \sub C(x, y) \times C(x, y) \}_{x, y \in C_0}$, we can construct a quotient category $C / R$ with equivalence classes as arrows.
There is a functor $Q : C \to C / R$ sending each arrow to its equivalence class, and for any functor $F : C \to D$ with $(f, g) \in R_{x, y}$ implies $F(f) = F(g)$,
there is a unique $F' : C / R \to D$ with $F = Q \fcmp F'$.
Intuitively, a functor from a quotient category $C / R$ is nothing more than a functor from $C$ which respects the axioms $R$.
Again, any small category $C$ is isomorphic to the quotient $F(\Sigma) / R$ of a free category $F(\Sigma)$: take $\Sigma = U(C)$ and $R = \{ (f \fcmp g, h) \in F(\Sigma) \times F(\Sigma) \s \vert \s f \fcmp g = h \in C \}$.
The pair $(\Sigma, R \sub \coprod_{x, y \in \Sigma_0} \Sigma(x, y) \times \Sigma(x, y))$ is called a presentation of the category $C \simeq F(\Sigma) / R$.
Since monoids are just categories with one object, the word problem for categories will be just as undecidable as for monoids.

What does it mean to implement a quotient category in Python?
Since presentations of categories are as expressive as Turing machines, we might as well avoid solving the halting problem and just use a Python function to define equality of arrows.
Implementing a quotient category is nothing more than implementing a free category and an equality function that respects the axioms of a congruence.
One straightforward way is to define equality of arrows $f, g$ in a free category $F(\Sigma)$ to be the equality of their interpretation $\eval{f} = \eval{g}$ under a functor $\eval{-} : F(\Sigma) \to D$ into a concrete category $D$ where equality is decidable.
Another method is to define a \emph{normal form} method which takes an arrow and returns the representative of its equivalence class, then identity of arrow is identity of their normal forms.

\begin{example}
Take the signature $\Sigma$ with one object $\Sigma_0 = \{ 1 \}$ and four arrows $\Sigma_1  = \{ Z, X, H, -1 \}$ for the Z, X and Hadamard gate and the global $(-1)$ phase.
Let's define the relation $R$ induced by:
\begin{itemize}
    \item $H X = Z H$ and $Z X = (-1) X Z$,
    \item $f f = 1$ and $f (-1) = (-1) f$ for all $f \in \Sigma_1$.
\end{itemize}
The quotient $F(\Sigma) / R$ is a subcategory of the category $\mathbf{Circ}$ of quantum circuits, it is isomorphic to the quotient induced by the interpretation $\eval{-} : F(\Sigma) \to \mathbf{Mat}_\C$.
Suppose we're given a functor $\mathtt{cost} : F(\Sigma) \to \R^+$, we can define the normal form of a circuit $f$ to be the representative of its equivalence class with the lowest cost.
Thus, deciding equality of circuits reduces to solving circuit optimisation perfectly.
\end{example}

We conclude this section by discussing three extra pieces of implementation beyond the basics of category theory: dagger, sums and bubbles.
A \emph{dagger} for a category $C$ can be thought of as a kind of time-reversal for arrows.
More precisely, a dagger is a contravariant endofunctor $\dagger : C \to C^{op}$, i.e. from the category to its opposite with $\dom$ and $\cod$ swapped, which is the identity on objects and an involution, i.e. $(\dagger) \fcmp (\dagger) = \id_\C$.
DisCoPy implements free $\dagger$-categories by adding an attribute \py{_dagger: bool} to boxes and a method \py{Arrow.dagger}, shortened to the postfix \py{[::-1]}, which reverses the order of boxes and negates \py{_dagger} elementwise.
A $\dagger$-functor is a functor between $\dagger$-categories which commutes with the dagger, they are implemented by adding a case to the code for functor application to define \py{F(box) = F(box[::-1])[::-1]} whenever \py{box._dagger} is true.
For example, the conjugate transpose defines a dagger on the category $\mathbf{Mat}_\S$, the adjoint defines a dagger on the category $\mathbf{Circ}$ and the evaluation $\mathbf{Circ} \to \mathbf{Mat}_\S$ is a $\dagger$-functor.

A category $C$ has \emph{sums}, or equivalently $C$ is \emph{monoid-enriched}, when it comes equipped with a commutative monoid $(+, 0)$ on each homset $C(x, y)$ such that $f \fcmp 0 = 0 = 0 \fcmp f$ and $(f + f') \fcmp (g + g') = f \fcmp g + f \fcmp g' + f' \fcmp g + f' \fcmp g'$ for all arrows $f, g, f', g'$.
A functor $F : C \to D$ between categories with sums is monoid-enriched when $F(0) = 0$ and $F(f + g) = F(f) + F(g)$.
For example, the category $\mathbf{Mat}_\S$ has sums given by elementwise addition of matrices.
In DisCoPy, free categories with sums are implemented by \py{Sum}, a subclass of \py{Box} with an attribute \py{terms: list[Arrow]}.
The method \py{then} is straightforward: the composition of a sum is the sum of the compositions of its terms.
Defining equality requires some extra care however: we want an arrow to be equal to the sum of just itself, we also want two sums to be equal when their list of terms are permutations of each other.
DisCoPy functors are monoid-enriched, i.e. formal sum of arrows can be interpreted as a concrete sum of matrices.

By a \emph{bubble} we mean an operator which takes an arrow in a category $C$ and puts it into a box.
More formally, a bubble is a pair of functions $b_\dom, b_\cod : C_0 \to C_0$ between objects and a unary operator between homsets $b : C(x, y) \to C(b_\dom(x), b_\cod(y))$ for each pair of objects $x, y \in C_0$.
DisCoPy implements the free category with bubbles via a \py{Bubble} class initialised by \py{dom, cod: Ob} and an attribute \py{inside: Arrow}.
DisCoPy functors automatically respect bubbles, i.e. we have that $F(b(f)) = b(F(f))$ for all arrows $f$.
Thus, we can interpret arrows with bubbles as arbitrary operations on the codomain of our interpretation functors.
For example, we can define a negation bubble on the category $\mathbf{Mat}_\B$ of Boolean matrices: it is the identity on objects and sends each matrix $f$ to its entrywise negation $\bar{f}$.
The resulting syntax with bubbles is strictly more expressive than that of free categories alone: negation cannot be expressed as a composition, there is no matrix $n$ in $\mathbf{Mat}_\B$ such that $f \fcmp n = \bar{f}$ for all matrices $f$.
As we will discuss in Chapter~\ref{chapter-3:diag-diff}, differentiation of parameterised matrices cannot be expressed as a composition either, but it is a unary operator between homsets, i.e. a bubble.

%!TEX root = ../../THESIS.tex

\section{String diagrams in Python}

In the previous section, we introduced the idea of arrows in free categories as formal data pipelines and functor application as their evaluation in concrete categories such as $\mathbf{Pyth}$, $\mathbf{Mat}$ or $\mathbf{Circ}$ where the computation happens.
For now, our pipelines are rather basic because they are linear: we cannot express functions of multiple arguments, nor tensors of order higher than 2, nor circuits with multiple qubits in any explicit way.
In this section, we move from the one-dimensional syntax of arrows in free categories to the two-dimensional syntax of \emph{string diagrams}, the arrows of free \emph{monoidal categories}.

%!TEX root = ../../THESIS.tex

\section{Drawing \& reading}

The previous section defined diagrams as a data structure based on lists of layers, in this section we define \emph{pictures of diagrams}.
Concretely, such a picture will be encoded in a computer memory as a bitmap, i.e. a matrix of colour values.
Abstractly, we will define these pictures in terms of topological subsets of the Cartesian plane.
We first recall the topological definition from Joyal's and Street's unpublished manuscript \emph{Planar diagrams and tensor algebra}~\cite{JoyalStreet88} and then discuss the isomorphism between the two definitions.
In one direction, the isomorphism sends a \py{Diagram} object to its drawing.
In the other direction, it reads the picture of a diagram and translates it into a \py{Diagram} object, i.e. its domain, codomain and list of layers.

\subsection{Labeled generic progressive plane graphs}

A \emph{topological graph}, also called 1-dimensional cell complex, is a tuple $(G, G_0, G_1)$ of a Hausdorff space $G$ and a pair of a closed subset $G_0 \sub G$ and a set of open subsets $G_1 \sub P(G)$ called \emph{nodes} and \emph{edges} respectively, such that:
\begin{itemize}
\item $G_0$ is discrete and $G - G_0 = \bigcup G_1$,
\item each edge $e \in G_1$ is homeomorphic to an open interval and its boundary is contained in the nodes $\partial e \sub G_0$.
\end{itemize}
From a topological graph $G$, one can construct an undirected graph in the usual sense by forgetting the space $G$, taking $G_0$ as nodes and edges $(x, y) \in G_0 \times G_0$ for each $e \in G_1$ with $\partial e = \{ x, y \}$.
A topological graph is finite (planar) if its undirected graph is finite (planar, i.e. there is some embedding in the plane).

A \emph{plane graph} between two real numbers $a < b$ is a finite, planar topological graph $G$ with an embedding in $\R \times [a, b]$.
We define the domain $\dom(G) = G_0 \ \cap \ \R \times \{ a \}$, the codomain $\cod(G) = G_0 \ \cap \ \R \times \{ b \}$ as lists of nodes ordered by horizontal coordinates and the set $\boxes(G) = G_0 \ \cap \ \R \times (a, b)$.
We require that:
\begin{itemize}
    \item $G \ \cap \ \R \times \{ a \} = \dom(G)$ and $G \ \cap \ \R \times \{ b \} = \cod(G)$, i.e. the graph touches the horizontal boundaries only at domain and codomain nodes,
    \item every domain and codomain node $x \in G \ \cap \ \R \times \{ a, b \}$ is in the boundary of exactly one edge $e \in G_1$, i.e. edges can only meet at box nodes.
\end{itemize}
A plane graph is \emph{generic} when the projection on the vertical axis $p_1 : \R \times \R \to \R$ is injective on $G_0 \ - \ \R \times \{ a, b \}$, i.e. no two box nodes are at the same height.
From a generic plane graph, we can get a list $\boxes(G) \in G_0^\star$ ordered by height.
A plane graph is \emph{progressive} (also called \emph{recumbent} by Joyal and Street) when $p_1$ is injective on each edge $e \in G_1$, i.e. edges go from top to bottom and do not bend backwards.

From a progressive plane graph $G$, one can construct a directed graph by forgetting the space $G$, taking $G_0$ as nodes and edges $(x, y) \in G_0 \times G_0$ for each $e \in G_1$ with $\partial e = \{ x, y \}$ and $p_1(x) < p_1(y)$.
We can also define the domain and the codomain of each box node $\dom, \cod : \boxes(G) \to G_1^\star$ with
$\dom(x) = \{ e \in G_1 \ \vert \partial e = \{ x, y \}, p_1(x) < p_1(y) \}$ the edges coming in from the top and
$\cod(x) = \{ e \in G_1 \ \vert \partial e = \{ x, y \}, p_1(x) > p_1(y) \}$ the edges going out to the bottom, these sets are linearly ordered as follows.
Take some $\epsilon > 0$ such that the horizontal line at height $p_1(x) - \epsilon$ crosses each of the edges in the domain.
Then list $\dom(x) \in G_1^\star$ in order of horizontal coordinates of their intersection points, i.e. $e < e'$ if $p_0(y) < p_0(y')$ for the projection $p_0 : \R \times \R \to \R$ and $y^{(')} = e^{(')} \cap \{ p_1(x) - \epsilon \} \times \R$. Symmetrically we define the list of codomain nodes $\cod(x) \in G_1^\star$ with a horizontal line at $p_1 + \epsilon$.

A \emph{labeling} of progressive plane graph $G$ by a monoidal signature $\Sigma$ is a pair of functions from edges to objects $\lambda : G_1 \to \Sigma_0$ and from boxes to boxes $\lambda : \boxes(G) \to \Sigma_1$ which commutes with the domain and codomain.
From an lgpp (\emph{labeled generic progressive plane}) graph, one can construct a \py{Diagram}.

\begin{python}
{\normalfont Translation from labeled generic progressive plane graphs to \py{Diagram}.}
\vspace{5pt}
\hrule
\vspace{-15pt}
\begin{flalign*}
\py{def} \s & \py{read(} \s G, \s \lambda : G_1 \to \py{Ty}, \s \lambda : \boxes(G) \to \py{Box} \s \py{) -> Diagram:}&&\\
& \py{dom = [} \s \lambda(e) \s \py{for} \s x \in \dom(G) \s \py{for} \s e \in G_1 \s \py{if} \s x \in \partial e \s \py{]}&&\\
& \py{boxes = [} \s \lambda(x) \s \py{for} \s x \in \boxes(G) \s \py{]}&&\\
& \py{offsets = [len(} \s G_1 \ \cap \ \{ p_0(x) \} \times \R \s \py{) for} \s x \in \boxes(G) \s \py{]}&&\\
& \py{return decode(dom, zip(boxes, offsets))} &&
\vspace{-10pt}
\end{flalign*}
\hrule
\end{python}

\subsection{From diagrams to graphs and back}

In the other direction, there are many possible ways to draw a given \py{Diagram} as a lgpp graph, i.e. to embed its graph into the plane.
Vicary and Delpeuch \cite{DelpeuchVicary18} give a linear-time algorithm to compute such an embedding with the following disadvantage: the drawing of a tensor $f \otimes g$ does not necessarily look like the horizontal juxtaposition of the drawings for $f$ and $g$.
For example, if we tensor an identity with a scalar, the edge representing the identity will wiggle around the node representing the scalar.
DisCoPy uses a quadratic-time drawing algorithm with the following design decision: we make every edge a straight line and as vertical as possible.
We first initialise the lgpp graph of the identity with a constant spacing between each edge, then for each layer we update the embedding so that there is enough space for the output edges of the box before we add it to the graph.

\begin{python}
{\normalfont Outline of \py{Diagram.draw} from \py{Diagram} to \py{PlaneGraph}.}

\begin{minted}{python}
Embedding = dict[Node, tuple[float, float]]
PlaneGraph = tuple[Graph, Embedding]

def draw(self: Diagram) -> PlaneGraph:
    graph = diagram2graph(self)
    def make_space(scan: list[Node], box: Box, offset: int) -> float:
        """ Update the graph to make space and return the left of the box. """
    box_nodes = [Node('box', box, -1, j) for j, box in enumerate(self.boxes)]
    dom_nodes = scan = [Node('dom', x, i, -1) for i, x in enumerate(self.dom)]
    position = {node: (i, -1) for i, node in enumerate(dom_nodes)}
    for j, (left, box, _) in enumerate(self.layers):
        box_node, left_of_box = Node('box', box, -1, j), make_space(scan, box, offset)
        position[box_node] = (left_of_box + max(len(box.dom), len(box.cod)) / 2, j)
        for kind, epsilon in (('dom', -.1), ('cod', .1)):
            for i, x in enumerate(getattr(box, kind)):
                position[Node(kind, x, i, j)] = (left_of_box + i, j + epsilon)
        box_cod_nodes = [Node('cod', x, i, j) for i, x in enumerate(box.cod)]
        scan = scan[:len(left)] + box_cod_nodes + scan[len(left @ box.dom):]
    for i, x in enumerate(self.cod):
        position[Node('cod', x, i, len(self))] = (position[scan[i]][0], len(self))
    return graph, position
\end{minted}
\end{python}

Note that when we draw the plane graph for a diagram, we do not usually draw the box nodes as points.
Instead, we draw them as boxes, i.e. a box node $x \in \boxes(G)$ is depicted as the rectangle with corners $(l, p_1(x) \pm \epsilon)$ and $(r, p_1(x) \pm \epsilon)$ for $l, r \in \R$ the left- and right-most coordinate of its domain and codomain nodes.
In this way, we do not need to draw the in- and out-going edges of the box node: they are hidden by the rectangle.
The only exceptions are \emph{spider boxes} where we draw the box node (the head) and its outgoing edges (the legs of the spider) as well as \emph{cup and cap boxes} where we do not draw the box node at all, only its two outgoing edges which are drawn as Bézier curves to look like cups and caps respectively.
Spiders, cups and caps will be discussed, and drawn, in section~\ref{section:extra structure}.

\begin{example} TODO
identity, box, composition, tensor and interchanger
\end{example}

\begin{example} TODO
Eckmann-Hilton
\end{example}

\begin{example}
The following spiral diagram is the cubic worst-case for interchanger normal form.
It is also the quadratic worst-case for drawing, at each layer of the first half we need to update the position of every preceding layer in order to make space for the output edges.

\begin{minted}{python}
x = Ty('x')
f, g = Box('f', Ty(), x @ x), Box('g', x @ x, Ty())
u, v = Box('u', Ty(), x), Box('v', x, Ty())

def spiral(length: int) -> Diagram:
    diagram, n = u, length // 2 - 1
    for i in range(n): diagram >>= Id(x ** i) @ f @ Id(x ** (i + 1))
    diagram >>= Id(x ** n) @ v @ Id(x ** n)
    for i in range(n): diagram >>= Id(x ** (n - i - 1)) @ g @ Id(x ** (n - i - 1))
    return diagram

for i in [1, 2, 3, 8]: spiral(8)[:i + 1].draw(to_tikz=True)
spiral(8).normal_form().draw(to_tikz=True)
\end{minted}
\begin{center}
\tikzfig{img/spiral/1},
\tikzfig{img/spiral/2},
\tikzfig{img/spiral/3}, ... \\
\vspace{5pt}
\tikzfig{img/spiral/8} $\quad \sim \quad$ \tikzfig{img/spiral/nf}
\end{center}
\end{example}

Next, we define the inverse translation \py{graph2diagram}.

\begin{python}
{\normalfont Translation from \py{PlaneGraph} to \py{Diagram}.}

\begin{minted}{python}
def graph2diagram(graph: Graph, position: Embedding) -> Diagram:
    dom = Ty(*[node.label for node in graph.nodes if node.kind == 'dom' and node.j == -1])
    boxes = [node.label for node in graph.nodes if node.kind == 'box']
    scan, offsets = [Node('dom', x, i, -1) for i, x in enumerate(dom)], []
    for j, box in enumerate(boxes):
        left_of_box = position[Node('dom', box.dom[0], 0, j)][0]\
            if box.dom else position[Node('box', box, -1, j)][0]
        offset = len([node for node in scan if position[node][0] < left_of_box])
        box_cod_nodes = [Node('cod', x, i, j) for i, x in enumerate(box.cod)]
        scan = scan[:offset] + box_cod_nodes + scan[offset + len(box.dom):]
        offsets.append(offset)
    return decode(dom, zip(boxes, offsets))
\end{minted}
\end{python}

\begin{theorem}\label{theorem:graph2diagram(self.draw())}
\py{graph2diagram(self.draw()) == self} for all \py{self: Diagram}.
\end{theorem}

\begin{proof}
By induction on \py{n = len(self.layers)}.
If \py{not len(self.layers)} we get that \py{dom == self.dom} and \py{boxes == offsets == []}.
If the theorem holds for \py{self}, it holds for \py{self >> Layer(left, box, right)}.
Indeed, we have:
\begin{itemize}
\item \py{dom == self.dom and boxes == self.boxes + [box]}
\item \py{(x, Node('cod', self.cod[i], i, n)) in graph for i, x in enumerate(scan)}
\end{itemize}
Moreover, the horizontal coordinates of the nodes in \py{scan} are strictly increasing,
thus we get the desired \py{offsets == self.offsets + [len(left)]}.
\end{proof}

A \emph{deformation} $h : G \to G'$ between two labeled plane graphs $G, G'$ is a continuous map $h : G \times [0, 1] \to \R \times \R$ such that:
\begin{itemize}
\item $h(G, t)$ is a plane graph for all $t \in [0, 1]$, $h(G, 0) = G$ and $h(G, 1) = G'$,
\item $x \in \boxes(G)$ implies $h(x, t) \in \boxes(h(G, t))$ for all $t \in [0, 1]$,
\item $h(G, t) \fcmp \lambda = \lambda$ for all $t \in [0, 1]$, i.e. the labels are preserved throughout.
\end{itemize}
A deformation is progressive (generic) when $h(G, t)$ is progressive (generic) for all $t \in [0, 1]$.
We write $G \sim G'$ when there exists some deformation $h : G \to G'$, this defines an equivalence relation.

\begin{theorem}\label{theorem:g2d then d2g}
For all lgpp graphs $G$, \py{Diagram.draw(graph2diagram(} $G$ \py{))} $\sim G$ up to generic progressive deformation.
\end{theorem}

\begin{proof}
By induction on the length of $\boxes(G)$.
If there are no boxes, $G$ is the graph of the identity and we can deform it so that each edge is vertical with constant spacing.
If there is one box, $G$ is the graph of a layer and we can cut it in three vertical slices with the box node and its outgoing edges in the middle.
We can apply the case of the identity to the left and right slices, for the middle slice we make the edges straight with a constant spacing between the domain and codomain.
Because $G$ is generic, we can cut a graph with $n > 2$ boxes in two horizontal slices between the last and the one-before-last box, then apply the case for layers and the induction hypothesis.
To glue the two slices back together while keeping the edges straight, we need to make space for the edges going out of the box.

This deformation is indeed progressive, i.e. we never bend edges we only make them straight.
It is also generic, i.e. we never move a box node past another.
\end{proof}

\begin{theorem}\label{theorem:d2g then g2d}
There is a progressive deformation $h : G \to G'$ between two lgpp graphs iff \py{graph2diagram(} $G$ \py{) == graph2diagram(} $G'$ \py{)} up to interchanger.
\end{theorem}

\begin{proof}
By induction on the number $n$ of \emph{coincidences}, the times at which the deformation $h$ fails to be generic, i.e. two or more boxes are at the same height.
WLOG (i.e. up to continuous deformation of deformations) this happens at a discrete number of time steps $t_1, \dots, t_n \in [0, 1]$.
Again WLOG at each time step there is at most two boxes at the same height, e.g. if there are two boxes moving below a third at the same time, we deform the deformation so that they move one after the other.
The list of boxes and offsets is preserved under generic deformation, thus if $n = 0$ then \py{graph2diagram(} $G$ \py{) == graph2diagram(} $G'$ \py{)} on the nose.
If $n = 1$, take \py{i: int} the index of the box for which the coincidence happens and \py{left: bool} whether it is a left or right interchanger, then \py{graph2diagram(} $G$ \py{).interchange(i, left) == graph2diagram(} $G'$ \py{)}.
Given a deformation with $n + 1$ coincidences, we can cut it in two time slices with $1$ and $n$ coincidences respectively then apply the cases for $n = 1$ and the induction hypothesis.

For the converse, a proof of \py{graph2diagram(} $G$ \py{) == graph2diagram(} $G'$ \py{)}, i.e. a sequence of $n$ interchangers, translates into a deformation with $n$ coincidences.
DisCoPy can output these proofs as videos using \py{Diagram.normalize} to iterate through the rewriting steps and \py{Diagram.to_gif} to produce a \py{.gif} file.
\end{proof}

\subsection{A natural isomorphism}

We have established an isomorphism between the class of lgpp graphs (up to progressive deformation) and the class of \py{Diagram} objects (up to interchanger).
It remains to show that this actually forms an isomorphism of monoidal categories.
That is for every monoidal signature $\Sigma$, there is a monoidal category $G(\Sigma)$ with objects $\Sigma_0^\star$ and arrows the equivalence classes of lgpp graphs with labels in $\Sigma$.
The domain and codomain of an arrow is given by the labels of the domain and codomain of the graph.
The identity $\id(x_1 \dots x_n)$ is the graph with edges $(i, a) \to (i, b)$ for $i \leq n$ and $a, b \in \R$ the horizontal boundaries.
The tensor of two graphs $G$ and $G'$ is given by horizontal juxtaposition, i.e. take $w = \max(p_0(G)) + 1$ the right-most point of $G$ plus a margin and set $G \otimes G' = G \cup \{ (p_0(x) + w, p_1(x)) \ \vert \ x \in G' \}$.
The composition $G \fcmp G'$ is given by vertical juxtaposition and connecting the codomain nodes of $G$ to the domain nodes of $G'$.
That is, $G \fcmp G' = s^+(G) \cup s^-(G') \cup E$ for $s^\pm(x) = \big( p_0(x), \frac{p_1(x) \pm (b - a)}{2} \big)$ and edges $s^+(\cod(G)_i) \to s^-(\dom(G')_i) \in E$ for each $i \leq \len(\cod(G)) = \len(\dom(G'))$.

The deformations for the unitality axioms are straightforward: there is a deformation $G \fcmp \id(\cod(G)) \sim G \sim \id(\dom(G)) \fcmp G$ which contracts the edges of the identity graph, the unit of the tensor is the empty diagram so we have an equality $G \otimes \id(1) = G = \id(1) \otimes G$.
The deformations for the associativity axioms are better described by the hand-drawn diagrams of Joyal and Street in figure~\ref{fig:assoc}.

\begin{figure}[H]
\centering
\includegraphics[scale=0.15]{img/tensor-assoc.png}
\qquad \quad \includegraphics[scale=0.15]{img/compos-assoc.png}
\caption{Deformations for the associativity of tensor and composition.}
\label{fig:assoc}
\end{figure}

The interchange law holds on the nose, i.e. $(G \otimes G') \fcmp (H \otimes H') = (G \fcmp H) \otimes (G' \fcmp H')$, as witnessed by figure~\ref{fig:interchange}, the hand-drawn diagram which is the result of both sides.

\begin{figure}[H]
\centering
\includegraphics[scale=0.075]{img/interchange.png}
\caption{The graph of the interchange law.}
\label{fig:interchange}
\end{figure}

Thus, we have defined a monoidal category $G(\Sigma)$.
Given a morphism of monoidal signatures $f : \Sigma \to \Gamma$, there is a functor $G(f) : G(\Sigma) \to G(\Gamma)$ which sends a graph to itself relabeled with $f \fcmp \lambda$, its image on arrows is given in listing~\ref{listing:G_of_f}.
Hence, we have defined a functor $G : \mathbf{Monsig} \to \mathbf{MonCat}$ which we claim is naturally isomorphic to the free functor $F : \mathbf{Monsig} \to \mathbf{MonCat}$ defined in the previous section.

\begin{python}\label{listing:G_of_f}
{\normalfont Implementation of the functor $G : \mathbf{Monsig} \to \mathbf{MonCat}$ on arrows.}

\begin{minted}{python}
SigMorph = tuple[dict[Ob, Ob], dict[Box, Box]]

def G(f: SigMorph) -> Callable[[Graph], Graph]:
    def G_of_f(graph: Graph) -> Graph:
        relabel = lambda node: Node('box', f[1][node.label], node.i, node.j)\
            if node.kind == 'box'\
            else Node(node.kind, f[0][node.label], node.i, node.j)
        return Graph(map(relabel, graph.edges))
    return G_of_f
\end{minted}
\end{python}

\begin{theorem}
There is a natural isomorphism $F \simeq G$.
\end{theorem}

\begin{proof}
From theorems~\ref{theorem:g2d then d2g} and \ref{theorem:d2g then g2d}, we have an isomorphism between \py{Diagram} and \py{PlaneGraph} given by \py{d2g = Diagram.draw} and \py{g2d = graph2diagram}, up to deformation and interchanger respectively.
Now define the image of $F$ on arrows \py{F = lambda f: Functor(ob=f[0], ar=f[1])}.
Given a morphism of monoidal signatures \py{f: SigMorph} we have the following two naturality squares in $\mathbf{Pyth}$.
\begin{center}
    \begin{tikzcd}
    \py{Diagram} \ar{r}{\py{d2g}} \ar{d}[']{\py{F(f)}} &
    \py{PlaneGraph} \ar{d}{\py{G(f)}}\\
    \py{Diagram} \ar{r}{\py{d2g}} &
    \py{PlaneGraph}
    \end{tikzcd}
    \qquad \text{and} \qquad
    \begin{tikzcd}
    \py{PlaneGraph} \ar{r}{\py{g2d}} \ar{d}[']{\py{G(f)}} &
    \py{Diagram} \ar{d}{\py{F(f)}}\\
    \py{PlaneGraph} \ar{r}{\py{g2d}} &
    \py{Diagram}
    \end{tikzcd}
\end{center}
\end{proof}

\subsection{Daggers, sums and bubbles}

daggers and asymmetry

drawing sums and equations

drawing bubbles

\subsection{Automatic diagram recognition}

reading diagrams from bitmaps

applications to automatic analysis of document layout \cite{BorosEtAl19}

%!TEX root = ../../THESIS.tex

\section{Adding extra structure} \label{section:extra structure}

\subsection{Rigidity: wire bending}

\subsection{Symmetry: wire swapping}

\subsection{Cartesian closed categories}

\subsection{Hypergraph categories}

%!TEX root = ../../THESIS.tex

\subsection{Rigid categories \& wire bending} \label{subsection:rigid}

In sections~\ref{section:cat} and \ref{section:monoidal} we discussed the fundamental notion of \emph{adjunction} with the example of free-forgetful functors.
The definition of left and right adjoints in terms of unit and counit natural transformations makes sense in $\mathbf{Cat}$, but it can be translated in the context of any monoidal category $C$.
An object $x^l \in C_0$ is the left adjoint of $x \in C_0$ whenever there are two arrows $\ttcup(x) : x^l \otimes x \to 1$ and $\ttcap(x) : 1 \to x \otimes x^l$ (also called counit and unit) such that:
\begin{itemize}
\item $\ttcap(x) \otimes x \ \fcmp \ x \otimes \ttcup(x) \s = \s \id(x)$,
\ctikzfig{img/rigid/snake-left}
\item $x^l \otimes \ttcap(x) \ \fcmp \ \ttcup(x) \otimes x^l \s = \s \id(x^l)$.
\ctikzfig{img/rigid/snake-right}
\end{itemize}
This is equivalent to the condition that the functor $x^l \otimes - : C \to C$ is the left adjoint of $x \otimes - : C \to C$.
Symmetrically, $x^r \in C_0$ is the right-adjoint of $x \in C_0$ if $x$ is its left adjoint.
We say that $C$ is \emph{rigid} (also called \emph{autonomous}) if every object has a left and right adjoint.
From this definition we can deduce a number of properties:
\begin{itemize}
    \item adjoints are unique up to isomorphism,
    \item adjoints are monoid anti-homomorphisms, i.e. $(x \otimes y)^l \simeq y^l \otimes x^l$ and $1^l \simeq 1$,
    \item left and right adjoints cancel, i.e. $(x^l)^r \simeq x \simeq (x^r)^l$,
\end{itemize}
We say that $C$ is strictly rigid whenever these isomorphisms are in fact equalities, again one can show that any rigid category is monoidally equivalent to a strict one.
One can also show that cups and caps compose by nesting:
\begin{itemize}
\item $\ttcup(x \otimes y) \s = \s y^l \otimes \ttcup(x) \otimes y \ \fcmp \ \ttcup(y)$,
\ctikzfig{img/rigid/nesting-cups}
\item $\ttcap(x \otimes y) \s = \s \ttcap(x) \ \fcmp \ x \otimes \ttcap(y) \otimes x^l$,
\ctikzfig{img/rigid/nesting-caps}
\item $\ttcup(1) \s = \s \ttcap(1) \s = \s \id(1)$, drawn as the equality of three empty diagrams.
\end{itemize}
The first two equations are drawn as diagrams in a non-foo monoidal category, i.e. with wires for composite types and explicit boxes for tensor.
This can be taken as an inductive definition, once we have defined the cups and caps for generating objects, we have defined them for all types.
Thus, we can take the data for a (strictly) rigid category $C$ to be that of a free-on-objects monoidal category together with:
\begin{itemize}
    \item a pair of unary operators $(-)^l, (-)^r : C_0 \to C_0$ on generating objects,
    \item and a pair of functions $\ttcup, \ttcap : C_0 \to C_1$ witnessing that $x^l$ and $x^r$ are the left and right adjoints of each generating object $x \in C_0$.
\end{itemize}
Diagrams in rigid categories are more flexible than monoidal categories: we can bend wires.
They owe their name to the fact that they are less flexible than \emph{pivotal categories}.
For any rigid category $C$, there are two contravariant endofunctors, called the left and right \emph{transpose} respectively.
They send objects to their left and right adjoints, and each arrow $f : x \to y$ to
\begin{center}
\tikzfig{img/rigid/transpose-left}
\hspace{50pt} and \hspace{50pt}
\tikzfig{img/rigid/transpose-right}
\end{center}
A rigid category $C$ is called \emph{pivotal} when it has a monoidal natural isomorphism $x^l \sim x^r$ for each object $x$, which implies that the left and right transpose coincide: we can rotate diagrams by 360 degrees.
We say $C$ is strictly piotal when this isomorphism is an equality.
This is the case for any rigid category $C$ with a dagger structure: the dagger of the cup (cap) for an object $x$ is the cap (cup) of its left-adjoint $x^l$.
When this is the case, $C$ is called $\dagger$-pivotal.
We say $C$ is strictly piotal when left and right transpose are equal.

\begin{example}
Recall from example~\ref{example:endofunctors are monoidal} that for any category $C$, the category $C^C$ of endofunctors and natural transformations is monoidal.
Its subcategory with endofunctors that have both left and right adjoints is rigid.
Its subcategory with endofunctors that have equal left and right adjoints is pivotal.
\end{example}

\begin{example}
$\mathbf{Tensor}_\S$ is $\dagger$-pivotal with left and right adjoints given by list reversal, cups and caps by the Kronecker delta $\ttcup(n)(i, j) = \ttcap(n)(i, j) = 1$ if $i = j$ else $0$.
Note that for tensors of order greater than 2, the \emph{diagrammatic transpose} defined in this way differs from the usual \emph{algebraic transpose}: the former reverses list order while the latter is the identity on objects.
\end{example}

\begin{example}
$\mathbf{Circ}$ is $\dagger$-pivotal with the preparation of the Bell state as cap and the post-selected Bell measurement as cup (both are scaled by $\sqrt{2}$).
The snake equations yield a proof of correctness for the (post-selected) quantum teleportation protocol.
\end{example}

\begin{example}\label{example:pregroups}
A discrete rigid category is a group: if the cups and caps are identities then they define an inverse for the tensor.
A rigid preordered monoid (i.e. a rigid category with at most one arrow between any two objects) is called a (quasi\footnote
{In his original definition~\cite{Lambek99}, Lambek also requires that pregroups are \emph{partial orders}, i.e. preorders with antisymmetry $x \leq y$ and $y \leq x$ implies $x = y$.
This implies that pregroups are strictly rigid, but also that they cannot be free on objects: $\ttcup(x) \otimes \id(x) : x \otimes x^l \otimes x \to x$ and $\id(x) \otimes \ttcap(x) : x \to x \otimes x^l \otimes x$ together would imply $x = x \otimes x^l \otimes x$.
}) \emph{pregroup}, their application to NLP will be discussed in section~\ref{section:NLP}.
A commutative pregroup is a (preordered) abelian group: left and right adjoints coincide with the multiplicative inverse.

Natural examples of non-free non-commutative pregroups are hard to come by.
One exception is the monoid of \emph{monotone unbounded functions} $\Z \to \Z$ with composition as multiplication and pointwise order.
The left adjoint of $f : \Z \to \Z$ is defined such that $f^l(m)$ is the minimum $n \in \Z$ with $m \leq f(n)$ and symmetrically $f^r(m)$ is the maximum $n \in \N$ with $f(n) \leq m$.
Extending Cayley's theorem from groups to pregroups, Buszkowski~\cite[Proposition~2]{Buszkowski01} proved that every pregroup $G$ is in fact isomorphic to a subpregroup (i.e. a monoidal subcategory) of monotone functions $G \to G$.
\end{example}

Any monoidal functor $F : C \to D$ between two rigid categories $C$ and $D$ preserves left and right adjoints up to isomorphism, we say it is strict when it preserves them up to equality.
Thus, we have defined a subcategory $\mathbf{RigidCat} \injects \mathbf{MonCat}$.
We define a \emph{rigid signature} $\Sigma$ as a monoidal signature where the generating objects have the form $\Sigma_0 \times \Z$.
We identify $x \in \Sigma_0$ with $(x, 0) \in \Sigma_0 \times \Z$ and define the left and right adjoints $(x, z)^l = (x, z - 1)$ and $(x, z)^r = (x, z + 1)$.
The objects $\Sigma_0$ are called \emph{basic types}, their iterated adjoints $\Sigma_0 \times \Z$ are called \emph{simple types}.
The integer $z \in \Z$ is called the \emph{adjunction number} of the simple type $(x, z) \in \Sigma_0 \times \Z$ by Lambek and Preller~\cite{PrellerLambek07} and its \emph{winding number} by Joyal and Street~\cite{JoyalStreet88}.
Again, a morphism of rigid signatures $f : \Sigma \to \Gamma$ is a pair of functions $f : \Sigma_0 \to \Gamma_0$ and $f : \Sigma_1 \to \Gamma_1$ which commute with domain and codomain.

There is a forgetful functor $U : \mathbf{RigidCat} \to \mathbf{RigidSig}$ which sends any strictly-rigid foo-monoidal category to its underlying rigid signature.
We now describe its left-adjoint $F^r : \mathbf{RigidSig} \to \mathbf{RigidCat}$.
Given a rigid signature $\Sigma$, we define a monoidal signature $\Sigma^r = \Sigma \cup \{ \ttcup (x) \}_{x \in \Sigma_0} \cup \{ \ttcap (x) \}_{x \in \Sigma_0}$.
The free rigid category is the quotient $F^r(\Sigma) = F(\Sigma^r) / R$ of the free monoidal category by the snake equations $R$.
That is, the objects are lists of simple types $(\Sigma_0 \times \Z)^\star$, the arrows are equivalence classes of diagrams with cup and cap boxes.
This is implemented in the \py{rigid} module of DisCoPy as outlined below.

\begin{python}
{\normalfont Implementation of objects and types of free rigid categories.}

\begin{minted}{python}
@dataclass
class Ob(cat.Ob):
    z: int

    l = property(lambda self: Ob(self.name, self.z - 1))
    r = property(lambda self: Ob(self.name, self.z + 1))

    @classmethod
    def cast(cls, old: cat.Ob) -> Ob:
        return old if isinstance(old, cls) else cls(str(old), z=0)

class Ty(monoidal.Ty, Ob):
    def __init__(self, inside=[]):
        monoidal.Ty.__init__(self, inside=map(Ob.cast, inside))

    l = property(lambda self: self.cast(Ty(*[x.l for x in self.inside[::-1]])))
    r = property(lambda self: self.cast(Ty(*[x.r for x in self.inside[::-1]])))
\end{minted}
\end{python}

\begin{example}
We can check the axioms for objects in rigid categories hold on the nose.

\begin{minted}{python}
x, y = Ty('x'), Ty('y')
assert Ty().l == Ty() == Ty().r
assert (x @ y).l == y.l @ x.l and (x @ y).r == y.r @ x.r
assert x.r.l == x == x.l.r
\end{minted}
\end{example}

\py{rigid.Ob} and \py{rigid.Ty} are implemented as subclasses of \py{cat.Ob} and \py{monoidal.Ty} respectively, with \py{property} methods (i.e. attributes that are computed on the fly) \py{l} and \py{r} for the left and right adjoints.
Thanks to the \py{cast} method, we do not need to override the \py{tensor} method inherited from \py{monoidal.Ty}.
In turn, subclasses of \py{rigid.Ty} will not need to override \py{l} and \py{r}.
Similarly, the \py{rigid.Diagram} class is a subclass of \py{monoidal.Diagram}, thanks to the \py{cast} we do not need to reimplement the identity, composition or tensor.
\py{rigid.Box} is a subclass of \py{monoidal.Box} and \py{rigid.Diagram}, with \py{Box.cast = Diagram.cast}.
We need to be careful with the order of inheritance however, diagram equality is defined in terms of box equality, so if we had \py{Box.__eq__ = Diagram.__eq__} then checking equality would enter an infinite loop.
\py{Cup} (\py{Cap}) is a subclass of \py{Box} initialised by a pair of types \py{x, y} such that \py{len(x) == len(y) == 1} \py{x == y.l} (\py{x.l == y}, respectively).
The class methods \py{cups} and \py{caps} construct diagrams of nested cups and caps by induction, with \py{Cup} and \py{Cap} as a base case.

\begin{python}
{\normalfont Implementation of the arrows of free rigid categories.}

\begin{minted}{python}
class Diagram(monoidal.Diagram):
    def transpose(self, left=True) -> Diagram:
        if left: ... # Symmetric to the right case.
        return self.caps(self.dom.r, self.dom) @ self.id(self.cod.r)\
            >> self.id(self.dom.r) @ self @ self.id(self.cod.r)\
            >> self.id(self.dom.r) @ self.cups(self.cod, self.cod.r)

class Box(monoidal.Box, Diagram):
    cast = Diagram.cast

class Cup(Box):
    def __init__(self, x: Ty, y: Ty):
        assert len(x) == 1 and x == y.l
        super().__init__("Cup({}, {})".format(x, y), x @ y, Ty())

class Cap(Box):
    def __init__(self, x: Ty, y: Ty):
        assert len(x) == 1 and x.l == y
        super().__init__("Cap({}, {})".format(x, y), Ty(), x @ y)

def nesting(factory):
    @classmethod
    def method(cls, x: Ty, y: Ty) -> Diagram:
        if len(x) == 0: return cls.id(Ty())
        if len(x) == 1: return factory(x, y)
        head = factory(x[0], y[-1])
        if head.dom:  # We are nesting cups.
            return x[0] @ method(x[1:], y[:-1]) @ y[-1] >> head
        return head >> x[0] @ method(x[1:], y[:-1]) @ y[-1]
    return method

Diagram.cups, Diagram.caps = nesting(Cup), nesting(Cap)
\end{minted}
\end{python}

The \emph{snake removal} algorithm listed below computes the normal form of diagrams in rigid categories.
It is a concrete implementation of the abstract algorithm described in pictures by Dunn and Vicary~\cite[2.12]{DunnVicary19}.
First, we implement a subroutine \py{follow_wire} takes a codomain node (given by the index \py{i} of its box and the index \py{j} of itself in the box's codomain)
follows the wire till it finds either the domain of another box or the codomain of the diagram.
When we follow a wire, we compute two lists of \emph{obstructions}, the index of each box on its left and right.
The \py{find_snake} function calls \py{follow_wire} for each \py{Cap} in the diagram until it finds one that is connected to a \py{Cup}, or returns \py{None} otherwise.
A \py{Yankable} snake is given by the index of its cup and cap, the two lists of obstructions on each side and whether it is a left or right snake.
\py{unsnake} applies \py{interchange} repeatedly to remove the obstructions, i.e. to make the cup and cap consecutive boxes in the diagram, then returns the diagram with the snake removed.
Each snake removed reduces the length $n$ of the diagram by 2, hence the \py{snake_removal} algorithm makes at most $n / 2$ calls to \py{find_snake}.
Finally, we call \py{monoidal.Diagram.normal_form} which takes at most cubic time.
Finding a snake takes quadratic time (for each cap we need to follow the wire at each layer) as well as removing it (for each obstruction we make a linear number of calls to \py{interchange}).
Thus, we can compute normal forms for diagrams in free rigid categories in cubic time.
We conjecture that we can in fact solve the word problem (i.e. deciding whether two diagrams are equal) in quadratic time using the same reduction to planar map isomorphism as in theorem~\ref{theorem:monoidal-normal-form}.

\begin{python}\label{listing:snake-removal}
{\normalfont Outline of the snake removal algorithm.}

\begin{minted}{python}
Obstruction = tuple[tuple[int, ...], tuple[int, ...]]
Yankable = tuple[int, int, Obstruction, bool]

def follow_wire(self: Diagram, i: int, j: int) -> tuple[int, int, Obstruction]: ...
def find_snake(self: Diagram) -> Optional[Yankable]: ...
def unsnake(self: Diagram, yankable: Yankable) -> Diagram: ...
def snake_removal(self: Diagram) -> Diagram:
    yankable = find_snake(diagram)
    return snake_removal(unsnake(diagram, yankable)) if yankable else diagram

Diagram.normal_form = lambda self:\
    monoidal.Diagram.normal_form(snake_removal(self))
\end{minted}
\end{python}

\begin{example}
We can check that the snake equations hold up to normal form.

\begin{minted}{python}
t = x @ y

left_snake = Diagram.id(t.l).transpose(left=False)
right_snake = Diagram.id(t).transpose(left=True)

assert left_snake.normal_form() == Diagram.id(t)\
    and right_snake.normal_form() == Diagram.id(t.l)

drawing.equation(
    drawing.Equation(left_snake, Diagram.id(t)),
    drawing.Equation(right_snake, Diagram.id(t.l)),
    symbol='and', space=2, draw_type_labels=False)
\end{minted}

\ctikzfig{img/rigid/double-snake-equation}
\end{example}

\begin{example}
We can check that left and right transpose cancel up to normal form.

\begin{minted}{python}
f = Box('f', x, y)

left_right_transpose = f.transpose(left=True).transpose(left=False)
right_left_transpose = f.transpose(left=False).transpose(left=True)

assert left_right_transpose.normal_form() == f == right_left_transpose.normal_form()
drawing.equation(left_right_transpose, f, right_left_transpose)
\end{minted}

\ctikzfig{img/rigid/transpose-inverse}
\end{example}

\begin{python}\label{example:pivotal-circuit}
{\normalfont Implementation of $\mathbf{Circ}$ as a pivotal category.}

\begin{minted}{python}
class Qubits(monoidal.Qubits, Ty):
    l = r = property(lambda self: self)

class Circuit(monoidal.Circuit, Diagram):
    cups = nesting(lambda _, _: sqrt2 @ Ket(0, 0) >> H @ qubit)
    caps = lambda x, y: Circuit.cups(x, y).dagger()
\end{minted}
\end{python}

\begin{example}
We can verify the teleportation protocol for two qubits.

\begin{minted}{python}
two_qubit_Bell_state = Circuit.caps(qubit ** 2)
two_qubit_Bell_effect = Circuit.cups(qubit ** 2)

assert (two_qubit_Bell_state @ qubit ** 2 >> qubit ** 2 @ two_qubit_Bell_effect).eval()\
    == (qubit ** 2).eval()\
    == (qubit ** 2 @ two_qubit_Bell_state >> two_qubit_Bell_effect @ qubit ** 2).eval()
\end{minted}
\end{example}

\py{rigid.Functor} is implemented as a subclass of \py{monoidal.Functor} with the \py{__call__} method overriden.
The image on types and on objects \py{x} with \py{x.z == 0} remains unchanged.
The image on objects \py{x} with \py{x.z < 0} is defined by \py{F(x) = F(x.r).l} and symmetrically for \py{x.z > 0}.
Indeed, when defining a strict rigid functor we only need to define the image of basic types, the image of their iterated adjoints is completely determined.
The only problem arises when the objects in the codomain do not have \py{l} and \py{r} attributes, such as the implementation of $\mathbf{Tensor}_\S$ with \py{list[int]} as objects.
In this case, we assume that the left and right adjoints are given by list reversal.

\begin{python}
{\normalfont Implementation of strict rigid functors.}
\begin{minted}{python}
class Functor(monoidal.Functor):
    dom = cod = Category(Ty, Diagram)

    def __call__(self, other):
        if isinstance(other, Ty) or isinstance(other, Ob) and other.z == 0:
            return super().__call__(other)
        if isinstance(other, Ob):
            if not hasattr(self.cod.ob, 'l' if other.z < 0 else 'r'):
                return self(Ob(other.name, z=0))[::-1]
            return self(other.r).l if other.z < 0 else self(other.l).r
        if isinstance(other, Cup):
            return self.cod.ar.cups(self(other.dom[:1]), self(other.dom[1:]))
        if isinstance(other, Cap):
            return self.cod.ar.caps(self(other.dom[:1]), self(other.dom[1:]))
        return super().__call__(other)
\end{minted}
\end{python}

\begin{python}
{\normalfont Implementation of $\mathbf{Tensor}_\S$ as a pivotal category.}
\begin{minted}{python}
Tensor.cups = classmethod(lambda cls, x, y: cls(cls.id(x).inside, x + y, []))
Tensor.caps = classmethod(lambda cls, x, y: cls(cls.id(x).inside, [], x + y))
\end{minted}
\end{python}

\begin{example}
We can check that $\mathbf{Tensor}_\S$ is indeed pivotal.

\begin{minted}{python}
F = Functor(
    ob={x: 2, y: 3}, ar={f: [[1, 2, 3], [4, 5, 6]]}
    cod=Category(tuple[int, ...], Tensor[int]))

assert F(left_snake) == F(Diagram.id(x)) == F(right_snake)
assert F(f.transpose(left=True)) == F(f).transpose() == F(f.transpose(left=False))

# Diagrammatic and algebraic transpose differ for tensors of order >= 2.
assert F(f @ f).transpose() != F((f @ f).transpose())
\end{minted}
\end{example}

Free pivotal categories are defined in a similar way to free rigid categories, with the two-element field $\Z / 2 \Z$ instead of the integers $\Z$, i.e. simple types with adjunction numbers of the same parity are equal.
In this case, we usually write $x^l = x^r = x^\star$ with $(x^\star)^\star = x$.
Given a pivotal signature $\Sigma$ with objects of the form $\Sigma_0 \times (\Z / 2 \Z)$, the free pivotal category is the quotient $F^p(\Sigma) = F^r(\Sigma) / R$ of the free rigid category by the relation $R$ equating the left and right transpose of the identity for each generating object.
While the diagrams of free rigid categories can have snakes, those of free pivotal categories can have circles: we can compose $\ttcap(x) : 1 \to x^l \otimes x$ then $\ttcup(x^\star) : x^l \otimes x \to 1$ to form a scalar diagram called the \emph{dimension} of the system $x$.
We also draw the wires with an orientation: the wire for $x$ is labeled with an arrow going down, the one for $x^\star$ with an arrow going up.

To the best of our knowledge, the word problem for pivotal categories is still open.
When defining the normal form of pivotal diagrams, we would need to make a choice between the diagrams for left or right transpose of a box.
Another solution is to add a new box $f^T : y^\star \to x^\star$ for the transpose of every box $f : x \to y$ in the signature, and set it as the normal form of both diagrams.
We can add some asymmetry to the drawing of the box $f$, and draw $f^T$ as its 180° degree rotation.
If the category is also $\dagger$-pivotal, we get a four-fold symmetry: the box, its dagger, its transpose and its dagger-transpose (also called its conjugate).
This is still being developed by the DisCoPy community.

\begin{python}
{\normalfont Implementation of free $\dagger$-pivotal categories.}

\begin{minted}{python}
class Ob(rigid.Ob):
    l = r = property(lambda self: self.cast(Ob(self.name, (self.z + 1) % 2)))

class Ty(rigid.Ty, Ob):
    def __init__(self, inside=[]):
        rigid.Ty.__init__(self, inside=map(Ob.cast, inside))

class Diagram(rigid.Diagram): pass

class Box(rigid.Box, Diagram):
    cast = Diagram.cast

class Cup(rigid.Cup, Box):
    def dagger(self):
        return Cap(self.dom[0], self.dom[1])

class Cap(rigid.Cap, Box):
    def dagger(self):
        return Cup(self.cod[0], self.cod[1])

Diagram.cups, Diagram.caps = nesting(Cup), nesting(Cap)

class Functor(rigid.Functor):
    dom = cod = Category(Ty, Diagram)
\end{minted}
\end{python}

%!TEX root = ../../THESIS.tex

\subsection{Braided categories \& wire crossing} \label{subsection:symmetric}

With rigid and pivotal categories, we have removed the assumption that diagrams are progressive: we can bend wires.
With braided and symmetric monoidal categories, we now remove the planarity assumption: wires can cross.

A monoidal category $C$ is \emph{braided} when it comes with a natural isomorphism $B(x, y) : x \otimes y \to y \otimes x$ subject to the following \emph{hexagon equations}:
\begin{itemize}
\item $B(x, y \otimes z) = B(x,y) \otimes z \s \fcmp \s y \otimes B(x,z)$
\item $B(x \otimes y, z) = x \otimes B(y,z) \s \fcmp \s B(x,z) \otimes y$
\end{itemize}
which owe their name to the shape of the corresponding commutative diagrams when $C$ is non-strict monoidal.
The box for the braiding $B(x, y)$ (its inverse $B^{-1}(x, y)$) is drawn as a wire for $x$ crossing under (over) a wire $y$.
We can draw the hexagon equations as non-free-on-objects diagrams, i.e. with explicit equality boxes:
\begin{center}
\tikzfig{img/symmetric/hexagon-left}\\
\vspace{10pt}
and \hspace{10pt} \tikzfig{img/symmetric/hexagon-right}
\end{center}
This can be taken as an inductive definition: the braiding $B(x,1)$ of an object with the unit $1$ is the identity, and we can decompose the braiding $B(x, y \otimes z)$ of an object with a tensor in terms of two simpler braids $B(x,y)$ and $B(x,z)$.
Thus, we can take the data for a braided category to be that of a foo-monoidal category together with a pair of functions $B, B^{-1} : C_0 \times C_0 \to C_1$ which send a pair of generating objects to their braiding and its inverse.
Once we have specified the braids of generating objects, the braids of any type (i.e. list of objects) is uniquely determined.
A monoidal functor $F : C \to D$ between two braided categories $C$ and $D$ is braided when $F(B(x, y)) = B(F(x), F(y))$, thus we get a category $\mathbf{BraidCat}$.

A braided category $C$ is \emph{symmetric} if the braiding $B$ is its own inverse $B = B^{-1} = S$, in this case it is called a \emph{swap}.
A symmetric functor is a braided functor between symmetric categories, thus we get a category $\mathbf{SymCat}$, with a forgetful functor $U : \mathbf{SymCat} \to \mathbf{MonSig}$.
We now describe its left adjoint.
Given a monoidal signature $\Sigma$, the free symmetric category is a quotient $F^s(\Sigma) = F(\Sigma \cup \{ S(x, y) \}_{x, y \in \Sigma_0}) / R$ of the free monoidal category generated by $\Sigma$ and the swaps $S$ for each pair of generating objects $x, y \in \Sigma_0$.
The relation $R$ is given by the following axioms for a self-inverse natural transformation:
\begin{itemize}
\item $S(x, y) \fcmp S(y, x) \s = \s \id(x \otimes y)$,
\item $f \otimes x \ \fcmp \ S(b, x) \s = \s S(a, x) \ \fcmp \ x \otimes f$,
\item $x \otimes f \ \fcmp \ S(x, b) \s = \s S(x, a) \ \fcmp \ f \otimes x$,
\end{itemize}
for all generating objects $x, y \in \Sigma_0$ and boxes (including swaps) $f : a \to b$ in $\Sigma_1 \cup \{ S(x, y) \}_{x, y \in \Sigma_0}$.
From $S$ being self-inverse on generating objects, we can prove it is self-inverse on any type by induction.
Similarly, from $S$ being natural on the left and right for each box, we can prove by induction that it is in fact natural for any diagram.
Drawing the box $S$ as a swap, i.e. two wires intersecting, we get the following diagrammatic equations:
\ctikzfig{img/symmetric/self-inverse}
\begin{center}
\tikzfig{img/symmetric/naturality-left}
\hspace{100pt}
\tikzfig{img/symmetric/naturality-right}
\end{center}
Note that the naturality axiom holds for boxes with domains and codomains of arbitrary length.
In particular, it holds for $f = S(y, z)$ in which case we get the following Yang-Baxter equation:
\ctikzfig{img/symmetric/yang-baxter}
It also holds for any scalar $f : 1 \to 1$, which allows to pass them through a wire:
\ctikzfig{img/symmetric/naturality-scalars}
A $\dagger$-braided category is a braided category with a dagger structure, such that the braidings are unitaries, i.e. their inverse is also their dagger.
A $\dagger$-symmetric category is a $\dagger$-braided category that is also symmetric.
DisCoPy implements free $\dagger$-symmetric ($\dagger$-braided) categories with a class \py{Swap} (\py{Braid}) initialised by types of length one and a class method \py{Diagram.swap} (\py{Diagram.braid}) for types of arbitrary length.

\begin{python}
{\normalfont Implementation of free $\dagger$-braided and $\dagger$-symmetric categories.}

\begin{minted}{python}
class Braid(monoidal.Box):
    def __init__(self, x: Ty, y: Ty, is_dagger=False):
        assert len(x) == len(y) == 1
        super().__init__("Braid({}, {})".format(x, y), x @ y, y @ x, is_dagger)

    def dagger(self): return Braid(self.dom[1], self.dom[0], not self.is_dagger)

class Swap(Braid):
    def __init__(self, x: Ty, y: Ty):
        super().__init__(x, y); self.name = self.name.replace("Braid", "Swap")

    def dagger(self): return Swap(self.dom[1], self.dom[0])

def hexagon(box_factory=Braid) -> Callable:
    @classmethod
    def method(cls, x: Ty, y: Ty) -> Diagram:
        if len(x) == 0: return cls.id(y)
        if len(x) == 1:
            if len(y) == 1: return box_factory(x[0], y[0])
            return method(cls, x, y[:1]) @ cls.id(y[1:])\
                >> cls.id(y[:1]) @ method(cls, x, y[1:])  # left hexagon equation.
        return cls.id(x[:1]) @ method(cls, x[1:], y)\
            >> method(cls, x[:1], y) @ cls.id(x[1:])  # right hexagon equation.
    return method

Diagram.braid, Diagram.swap = hexagon(), hexagon(Swap)

class Functor(monoidal.Functor):
    def __call__(self, other):
        if isinstance(other, Swap):
            return self.ar_factory.swap(self(other[0]), self(other[1]))
        if isinstance(other, Braid) and not other.is_dagger:
            return self.ar_factory.braid(self(other[0]), self(other[1]))
        return super().__call__(other)
\end{minted}
\end{python}

\begin{example}
We can check the hexagon equations hold on the nose.

\begin{minted}{python}
x, y, z = map(Ty, "xyz")

assert Diagram.swap(x, y @ z) == Swap(x, y) @ z >> y @ Swap(x, z)
assert Diagram.swap(x @ y, z) == x @ Swap(y, z) >> Swap(x, z) @ y
\end{minted}
\end{example}

\begin{python}
{\normalfont Implementation of $\mathbf{Pyth}$ and $\mathbf{Tensor}_\S$ as symmetric categories.}

\begin{minted}{python}
def function_swap(x: list[type], y: list[type]) -> Function:
    def inside(*xs):
        assert len(xs) == len(x + y)
        return untuple(xs[len(x):] + xs[:len(x)])
    return Function(inside, dom=x + y, cod=y + x)

Function.swap = Function.braid = function_swap

@classmethod
def tensor_swap(cls, x: list[int], y: list[int]) -> Tensor:
    inside = [[(i0, j0) == (i1, j1)
        for j0 in range(product(y)) for i0 in range(product(x))]
        for i1 in range(product(x)) for j1 in range(product(y))]
    return cls(inside, dom=x + y, cod=y + x)

Tensor.swap = Tensor.braid = tensor_swaps
\end{minted}
\end{python}

Note that the naturality axioms in $\mathbf{Pyth}$ hold only for its subcategory of pure functions, as we will see in section~\ref{section:premonoidal} $\mathbf{Pyth}$ is in fact a symmetric \emph{premonoidal} category.
This is also the case for $\mathbf{Tensor}_\S$ when the rig $\S$ is non-commutative.

\begin{example}
We can check the axioms for self-inverse natural transformations hold in $\mathbf{Tensor}_\S$ and $\mathbf{Pyth}$.

\begin{minted}{python}
swap_twice = Diagram.swap(x, y @ z) >> Diagram.swap(y @ z, x)

a, b = Ty('a'), Ty('b')
f = Box('f', a, b)

F = Functor(
    ob={a: 2, b: 3, x: 4, y: 5, z: 6},
    ar={f: [[1-2j, 3+4j]]},
    ob_factory=list[int], ar_factory=Tensor[complex])

assert F(swap_twice) == Tensor.id(F(x @ y @ z))
assert F(f @ x >> Swap(b, x)) == F(Swap(a, x) >> x @ f)
assert F(x @ f >> Swap(x, b)) == F(Swap(x, a) >> f @ x)

G = Functor(
    ob={a: complex, b: real, x: int, y: bool, z: str},
    ar={f: lambda z: abs(z) ** 2},
    ob_factory=list[type], ar_factory=Function)

assert G(swap_twice)(42, True, "meaning of life") == (42, True, "meaning of life")
assert G(f @ x >> Swap(b, x))(1j, 2) == G(Swap(a, x) >> x @ f)(1j, 2) == (2, f(1j))
assert G(x @ f >> Swap(x, b))(2, 1j) == G(Swap(x, a) >> f @ x)(2, 1j) == (f(1j), 2)
\end{minted}
\end{example}

Free braided categories are defined in a similar way, with generating boxes for both the braid $B$ and its inverse $B^{-1}$.
The arrows of the free symmetric PRO (i.e. with one generating object $x$) with no generating boxes (i.e. only swaps) are called \emph{permutations}, the arrows of the free braided PRO with no boxes are called \emph{braids}.
Both are \emph{groupoids}, i.e. all their arrows are isomorphisms.
For every $n \in \N$, the arrows $f : x^n \to x^n$ in the free symmetric (braided) PRO form the $n$-th symmetric group $S_n$ (braid group $B_n$).

A \emph{compact closed category} is one that is both rigid and symmetric, which implies that it is also pivotal, a $\dagger$-compact closed category is both $\dagger$-pivotal and $\dagger$-symmetric.
The arrows of free $\dagger$-compact closed categories (i.e. equivalence classes of diagrams with cups, caps and swaps) are also called \emph{tensor networks}, a graphical equivalent to \emph{Einstein notation} and \emph{abstract index notation}, first introduced by Penrose~\cite{Penrose71}.
Unlike the computer scientists however, physicists tend to identify the diagram (syntax) with its image under some interpretation functor to the category of tensors (semantics).

A \emph{tortile category}, also called a \emph{ribbon category}\footnote
{Here again we take a \emph{strict} definition, where the twist is an identity rather than an isomorphism.
In a non-strict tortile category, the wires would be drawn as ribbons, i.e. two wires side by side.
The twist isomorphism would be drawn as the two wires being braided twice.
In a strict tortile category, the ribbon has no width thus the twist is invisible.},
is a braided, pivotal category which furthermore satisfies the following \emph{untwisting} equation:
\ctikzfig{img/symmetric/untwisting}
The scalars of the free tortile category with no boxes (i.e. equivalence classes of diagrams with only cups, caps and braids) are called \emph{links} in general and \emph{knots} when they are connected.
Untwisting, the self-inverse equation and the Yang-Baxter equation (i.e. naturality with respect to braids) are called the three \emph{Reidemeister moves}, they completely characterise the continuous deformations of circles embedded in three-dimensional space~\cite{Reidemeister13}.

The \emph{unknotting problem} (given a knot, can it be untied, i.e. continuously deformed to a circle?) is a candidate NP-intermediate problem: it is decidable~\cite{Haken61} and in NP~\cite{Lackenby15}, but there is neither a proof of it being NP-complete nor a polynomial-time algorithm.
Delpeuch and Vicary~\cite{DelpeuchVicary21} proved that the word problem for free braided categories is unknotting-hard.
Hence, there is little hope of finding a simple polynomial-time algorithm for computing normal forms of braided diagrams.
Guiraud and Malbos~\cite{GuiraudMalbos12} do give a proof that the word problem for free braided categories is in fact decidable, however they do not give any complexity bounds.

The word problem for free symmetric categories reduces to the \emph{graph isomorphism problem}~\cite{PattersonEtAl21}, another potential NP-intermediate problem.
The word problem for free compact closed categories also reduces to graph isomorphism~\cite{Selinger07}.
To the best of our knowledge, it is not known whether they are graph-isomorphism-hard, i.e. whether there is a reduction the other around that sends any graph to a diagram with swaps (and cups and caps) so that graphs are isomorphic if and only their diagrams are equal.
Thus, there could be a simple polynomial-time algorithm for computing normal forms of diagrams in symmetric and compact closed categories.
In any case, DisCoPy does not implement any normal forms for diagrams with braids yet.

Of course, we can also enrich rigid and braided categories in commutative monoids, i.e. we can take formal sums of diagrams with cups, caps and braids in the same way as any other box.
We can also define bubbles and draw them in the same way as for monoidal diagrams.

\begin{example}
We can define knot polynomials such as the Kauffman bracket using pivotal functors into a category where braids are defined as a weighted sum of diagrams.

\begin{minted}{python}
x = pivotal.Ty('$x$')
A, A.inverse = Box('$A$', Ty(), Ty()), Box('$A^{-1}$', Ty(), Ty())

class Polynomial(pivotal.Diagram):
    def braid(x, y):
        assert x == y and len(x) == len(y) == 1
        return (A @ x @ y) + (Cup(x, y) >> A.inverse >> Cap(x, y))

Kauffman = Functor(
    ob={x: x}, ar={}
    ob_factory=pivotal.Ty, ar_factory=Polynomial)

drawing.equation(Braid(x, x).bubble(), Kauffman(Braid(x, x)))
\end{minted}

\ctikzfig{img/symmetric/kauffman-bracket}
\end{example}

%!TEX root = ../../THESIS.tex

\subsection{Hypergraph categories \& wire splitting} \label{subsection:hypergraph}

With compact closed and tortile categories, we have removed both the progressivity and the planarity assumptions: wires can bend and cross.
With \emph{hypergraph categories} we remove the assumption that diagrams are graphs: wires can split and merge, they need not be homeomorphic to an open interval.
A hypergraph category is a symmetric category with \emph{coherent special commutative spiders}, let's spell out what this means.

An object $x$ in a monoidal category $C$ has \emph{spiders} with \emph{phases} in a monoid $(\Phi, +, 0)$ if it comes equipped with a family of arrows $\spider_{\phi, a, b}(x) : x^a \to x^b$ for every phase $\phi \in \Phi$ and pair of natural numbers $a, b \in \N$, such that the following \emph{spider fusion} equation holds for all $a, b, c, d, n \in \N$.
$$\spider_{\phi, a, c + n + 1}(x) \otimes x^b
\ \fcmp \ x^c \otimes \spider_{\phi', c + n + 1, d}(x)
\s = \s \spider_{\phi + \phi', a + b, c + d}(x)$$
We also require that our spiders satisfy the \emph{special} condition $\spider_{0, 1, 1}(x) = \id(x)$.
Spiders owe their name to their arachnomorphic drawing, for example $\spider_{\phi, 2, 6}$ is drawn as a node (the head, labeled by its phase when it's non-zero) and its wires (the eight legs of the spider, two of them menacing us):
\ctikzfig{img/hypergraph/spider}
Once drawn, the spider fusion equation has the intuitive graphical meaning that if one or more legs of two spiders touch, they fuse and add up their phase.
\ctikzfig{img/hypergraph/spider-fusion}
From spider fusion, we can deduce the following properties:
\begin{itemize}
\item $\merge(x) = \spider_{0, 2, 1}(x)$ and $\unit(x) = \spider_{0, 0, 1}(x)$ form a monoid,
\begin{center}
\tikzfig{img/hypergraph/assoc}
\hfill
\tikzfig{img/hypergraph/unit}
\end{center}
\item $\ttsplit(x) = \spider_{0, 1, 2}(x)$ and $\counit(x) = \spider_{0, 1, 0}(x)$ form a comonoid,
\begin{center}
\tikzfig{img/hypergraph/coassoc}
\hfill
\tikzfig{img/hypergraph/counit}
\end{center}
\item $\ttsplit(x) \fcmp \merge(x) = \id(x)$, called the \emph{special} condition,
\ctikzfig{img/hypergraph/special}
\item $\merge(x) \otimes x \fcmp x \otimes \ttsplit(x) \s = \s x \otimes \merge(x) \fcmp \ttsplit(x) \otimes x$, called the \emph{Frobenius law}.
\ctikzfig{img/hypergraph/frobenius}
\end{itemize}
In fact, when the phases are trivial $\Phi = \{ 0 \}$ these four axioms are sufficient to deduce spider fusion, spiders are also called \emph{special Frobenius algebras}.
Indeed, given a monoid $\merge(x) : x \otimes x \to x, \s \unit(x) : 1 \to x$ and a comonoid $\ttsplit(x) : x \to x \otimes x, \s \counit(x) : x \to 1$ subject to the Frobenius law, we can construct $\spider_{a, b}(x) : x^a \to x^b$ by induction on the number of legs.
The base case is given by the special condition $\spider_{1, 1}(x) = \id(x)$.
Then we define spiders with $a \in \N$ input legs for $a \neq 1$:
\begin{itemize}
\item $\spider_{0, b}(x) \s = \s \unit(x) \fcmp \spider_{1, b}(x)$,
\item $\spider_{a + 2, b}(x) \s = \s \merge(x) \otimes x^a \ \fcmp \ \spider_{a + 1, b}(x)$,
\end{itemize}
\begin{center}
\tikzfig{img/hypergraph/induction-base}
\hfill
\tikzfig{img/hypergraph/induction-step}
\end{center}
Finally we define spiders with one input leg by induction on the output legs $b \in \N$:
\begin{itemize}
\item $\spider_{1, 0}(x) \s = \s \counit(x)$,
\item $\spider_{1, b + 2}(x) \s = \s \spider_{1, b + 1}(x) \ \fcmp \ \ttsplit(x) \otimes x^b$.
\ctikzfig{img/hypergraph/induction-step-one-legged}
\end{itemize}
One can show that this satisfies the spider fusion law, again by induction on the legs~\cite[Lemma 5.20]{HeunenVicary19a}.
In this way, we can construct an infinite family of spiders from just the four boxes $\merge(x), \unit(x), \ttsplit(x), \counit(x)$ and a finite set of equations: a spider is nothing but a big multiplication followed by a big co-multiplication.
As for the phases, we can recover them from a family of \emph{phase shifts} $\{ \shift_\phi(x) : x \to x \}_{\phi \in \Phi}$ such that:
\begin{itemize}
\item $\shift_-(x)$ is a monoid homomorphism $\Phi \to C(x, x)$, i.e. $\shift_0(x) = \id(x)$ and $\shift_{\phi}(x) \fcmp \shift_{\phi'} = \shift_{\phi + \phi'}(x)$,
\ctikzfig{img/hypergraph/phase-hom}
\item phase shifts commute with the product, $\shift_\phi(x) \otimes x \ \fcmp \ \merge(x)
\ = \ \merge(x) \fcmp \shift_\phi(x)
\ = \ x \otimes \shift_\phi(x) \ \fcmp \ \merge(x)$,
\ctikzfig{img/hypergraph/phase-commute-product}
\item phase shifts commute with the coproduct, $\ttsplit(x) \ \fcmp \ \shift_\phi(x) \otimes x
\ = \ \shift_\phi(x) \fcmp \ttsplit(x)
\ = \ x \otimes \ttsplit(x) \ \fcmp \ \shift_\phi(x)$.
\ctikzfig{img/hypergraph/phase-commute-coproduct}
\end{itemize}
We can then define $\spider_{\phi, a, b}(x) = \spider_{a, 1}(x) \fcmp \shift_\phi(x) \fcmp \spider_{1, b}(x)$ and check that indeed, spiders fuse up to addition of their phase.
Thus when the monoid is finite, we get a finite number of boxes and equations, i.e. a finite presentation of the spiders.
In fact instead of taking it as data, we could have equivalently defined the monoid of phases $\Phi$ as the set of endomorphisms $x \to x$ that satisfy the last two conditions.

\begin{remark}
Given any Frobenius algebra on an object $x$, we can show that $x$ is its own left and right adjoint.
Indeed, take $\ttcup(x) = \unit(x) \fcmp \ttsplit(x)$ and $\ttcap(x) = \merge(x) \fcmp \counit(x)$, then the Frobenius law and the (co)unit law of the (co)monoid implies the snake equations.
Thus, a category with (not-necessarily special) spiders on every object is automatically a pivotal category.

\ctikzfig{img/hypergraph/spider-implies-snake}
\end{remark}

\begin{example}
In any pivotal category, there is a Frobenius algebra for every object of the form $x^\star \otimes x$ given by:
\begin{itemize}
\item $\merge(x^\star \otimes x) = x^\star \otimes \ttcup(x^\star) \otimes x$ and $\unit(x) = \ttcap(x)$,
\item $\ttsplit(x^\star \otimes x) = x^\star \otimes \ttcap(x) \otimes x$ and $\counit(x) = \ttcup(x^\star)$.
\end{itemize}
\ctikzfig{img/hypergraph/pair-of-pants}
Due to the drawing of its comonoid, this is called the \emph{pair of pants} algebra.
The special condition requires the dimension of the system $x$ to be the unit, i.e. the circle is equal to the empty diagram.
Non-special Frobenius algebras can still be drawn as spiders, they satisfy a modifed version of spider fusion where we keep track of the number of circles, i.e. the number of splits followed by a merge.
We can extend our inductive definition so that all the circles are in between the product and coproduct, see~\cite[Theorem 5.21]{HeunenVicary19a}.
\end{example}

\begin{example}\label{example:tensor-spider}
The category $\mathbf{Tensor}_\S$ has spiders for every dimension $n \in \N$ with phases in any submonoid of $\phi \in (\S, \times, 1)^n$.
They are given by $\spider_{\phi, a, b}(n) = \sum_{i \leq n} \phi_i \ket{i}^{\otimes a} \bra{i}^{\otimes b}$ where $\ket{i}$ ($\bra{i}$) is the $i$-th basis row (column) vector.

\begin{minted}{python}
class Tensor:
    ...
    @classmethod
    def spider(cls, a: int, b: int, n: int, phase=None) -> Tensor:
        phase = phase or n * [1]
        inside = [[sum(phase)]] if not a and not b\
            else [[phase[xs[0]] for xs in itertools.product(*b * [range(n)])
                   if all(x == xs[0] for x in xs)]]\
            if not a else cls.spider([], a + b, n).inside
        return cls(inside, dom=a * [n], cod=b * [n])
\end{minted}

When $\S$ is a field, we can divide every $\phi_i$ by $\phi_0$, or equivalently require that $\phi_0 = 1$.
Indeed, we can represent any spider with $\phi_0 \neq 1$ as a spider with $\phi_0 = 1$ multiplied by the scalar $\phi_0$, which is called a \emph{global phase}.
When $\S = \C$ and $n = 2$, we usually take the monoid of phases to be the unit circle and write it in terms of addition of angles.
\end{example}

\begin{example}\label{example:circuit-spider}
In the category $\mathbf{Circ}$ of quantum circuits, if we allow post-selected measurements then we can construct spiders with the unit circle as phases.
The spiders with no inputs legs are called the (generalised) GHZ states:
$$
\spider_{\alpha, 0, b} = \ket{0}^{\otimes b} + e^{i \alpha} \ket{1}^{\otimes b}
$$
Note that we need to scale by $\frac{1}{\sqrt{2}}$ to make this a normalised quantum state.
The spiders with $a > 0$ input legs can be thought of as measuring $a$ qubits, post-selecting on all of them giving the same result and then preparing $b$ copies of this result.
The evaluation functor $\mathbf{Circ} \to \mathbf{Tensor}_\C$ sends spiders to spiders.
\end{example}

Spiders allow us to draw diagrams where wires can split and merge, connecting an arbitrary number of boxes.
The PRO of Frobenius algebras (without the special condition), i.e. diagrams with only spider boxes, defines a notion of ``well-behaved'' 1d subspaces of the plane, up to continuous deformation.
Indeed, it is equivalent to the category of \emph{planar thick tangles}~\cite{Lauda05}.
Intuitively, planar thick tangles can be thought of as planar wires with a width, i.e. that we can draw with pens or pixels.
The inductive definition of spiders in terms of monoids and comonoids has the topological interpretation that any wire can be deformed so that all its singular points (i.e. where the wire crosses itself) are binary splits and merges.
The special condition has the non-topological consequence that we can contract the holes in the wires, splitting a wire then merging it back does nothing.

If the monoidal category $C$ is braided, we can remove the planarity assumption and define \emph{commutative spiders} as those where the monoid and comonoid are commutative, i.e.
\begin{align*}
\spider_{\phi, a + b, c + d}(x) \ \fcmp \ B(x^c, x^d)
\s &= \s \spider_{\phi, a + b, c + d}(x) \\
\s &= \s
B(x^a, x^b) \ \fcmp \ \spider_{\phi, a + b, c + d}(x)
\end{align*}
\ctikzfig{img/hypergraph/co-commutativity}
Together with spider fusion, this implies that the monoid of phases is also commutative.
The PROB of commutative Frobenius algebras (without the special condition), i.e. diagrams with only spiders and braids, defines a notion of ``well-behaved'' 1d subspaces of 3d space, up to continuous deformation.
When the category is furthermore symmetric, the PROP of commutative spiders defines a notion of ``well-behaved'' 1d spaces up to diffeomorphism, or equivalently 1d subspaces of 4d space, i.e. one where wires can pass through each other and all knots untie.
It is equivalent to the category of two-dimensional \emph{cobordisms}~\cite{Abrams96}, i.e. oriented 2d manifolds with a disjoint union of circles as boundary.
Intuitively, a 2d cobordism can be thought of as a (non-planar) wire with a width, i.e. one that we can draw.

When $C$ is braided, we can also give an inductive definition of spiders for tensors.
Indeed, given the spiders for $x$ and $y$ we can construct the following comonoid:
\begin{itemize}
\item $\spider_{1, 0}(x \otimes y) \s = \s \spider_{1, 0}(x) \otimes \spider_{1, 0}(y)$,
\ctikzfig{img/hypergraph/coherence-unit}
\item $\spider_{1, 2}(x \otimes y) \s = \s \spider_{1, 2}(x) \otimes \spider_{1, 2}(y) \ \fcmp \ x \otimes S(x, y) \otimes y$
\ctikzfig{img/hypergraph/coherence-product}
\end{itemize}
and construct a monoid in a symmetric way, then show that they satisfy the spider fusion equations for $x \otimes y$.
We can also show that the identity of the unit defines a family of spiders, i.e. $\spider_{a, b}(1) = \id(1)$.
If we take them as axioms rather than definitions, these are called the \emph{coherence conditions} for spiders.

Thus we get to our definition: a \emph{hypergraph category} is a symmetric category with coherent special commutative spiders on each object.
We can take the data to be that of a foo-monoidal category $C$ together with a function $\spider : \N \times \N \times C_0 \to C_1$ or equivalenty, with four functions $\merge, \unit, \ttsplit, \counit : C_0 \to C_1$.
Once we fix the spiders for generating objects, we get spiders for any type (i.e. list of objects).
A hypergraph functor is a symmetric functor $F : C \to D$ between hypergraph categories such that $F \fcmp \spider_{a, b} = \spider_{a, b} \fcmp F$.
Thus we get a category $\mathbf{HypCat}$ with a forgetful functor $U : \mathbf{HypCat} \to \mathbf{MonSig}$.
Its left adjoint $F^H : \mathbf{MonSig} \to \mathbf{HypCat}$ is defined as a quotient $F^S(\Sigma^H) / R$ of the free symmetric category generated by $\Sigma^H = \Sigma \spider$ and the relation $R$ given by the equations for commutative spiders.
Equivalently, we can take $\Sigma^H = \bigcup \{ \Sigma, \merge, \unit, \ttsplit, \counit \}$ and $R$ given by the equations for special commutative Frobenius algebras.
A \emph{$\dagger$-hypergraph category} is a $\dagger$-symmetric category (i.e. the swaps are unitaries) where the dagger is a hypergraph functor.
We also require that the monoid of phases is in fact a group with the dagger as inverse or equivalently, that phase shifts are unitaries.

\begin{example}
For evert commutative rig $\S$, the category $\mathbf{Tensor}_\S$ is $\dagger$-hypergraph with the transpose as dagger.
Arguably, special commutative Frobenius algebras were first defined by Peirce~\cite{Peirce06} with their interpretation in the category of relations, or equivalently $\mathbf{Tensor}_\B$.
Indeed, they correspond to what Peirce calls \emph{lines of identity}: they express in two dimensions what one-dimensional first-order logic would express with equality symbols.
For example, take a binary predicate encoded as a box $p : 1 \to x^2$ (interpreted as the formula $\exists \ a \cdot \exists \ b \cdot p(a, b)$) then the diagram $p \fcmp \merge(x)$ is interpreted as the formula $\exists \ a \cdot \exists \ b \cdot p(a, b) \land a = b$ or equivalently $\exists \ a \cdot p(a, a)$.
Thus, every first-order logic formula can be written as a diagram with boxes for predicates, spiders for identity and bubbles for negation.
The equivalence of formulae can be defined as a quotient of a free hypergraph category with bubbles, i.e. all the rules of first-order logic can be given in terms of diagrams.
\end{example}

\begin{example}
The category of complex tensors $\mathbf{Tensor}_\C$ is $\dagger$-hypergraph with the spiders given in example~\ref{example:tensor-spider}.
Any unitary matrix $U : n \to n$ defines another family of spiders $U^{\otimes a} \fcmp \spider_{\phi, a, b}(n) \fcmp (U^\dagger)^{\otimes b}$.
In fact, every unitary arises in this way, see Heunen and Vicary~\cite[Corollary 5.32]{HeunenVicary19a}.
Thus, the axioms for spiders allow us to define any orthonormal basis without ever mentioning basis vectors: they are merely the states $v : 1 \to n$ for which the comonoid is natural, i.e. $v \fcmp \ttsplit(x) = v \otimes v$ and $v \fcmp \counit(x) = \id(1)$.
\end{example}

\begin{example}
The category $\mathbf{Circ}$ is $\dagger$-hypergraph with the spiders defined in example~\ref{example:circuit-spider}, the evaluation functor $\mathbf{Circ} \to \mathbf{Tensor}_\C$ is a $\dagger$-hypergraph functor.
\end{example}

DisCoPy implements spiders for types of length one (i.e. generating objects) as a subclass of \py{Box} and spiders for arbitrary types as a method \py{Diagram.spiders}.

\begin{python}
{\normalfont Implementation of $\dagger$-hypergraph categories and functors.}

\begin{minted}{python}
class Spider(Box):
    def __init__(self, a: int, b: int, x: Ty, phase=None):
        assert len(x) == 1
        self.object, self.phase = x, phase or 0
        name = "Spider({})".format(', '.join(map(str, (a, b, x, phase))))
        super().__init__(name, dom=x ** a, cod=x ** b)

    def dagger(self):
        a, b, x = len(self.cod), len(self.dom), self.object
        phase = None if self.phase is None else -self.phase
        return Spider(a, b, x, phase)

def coherence(factory):
    @classmethod
    def method(cls, a: int, b: int, x: Ty, phase=None) -> Diagram:
        if len(x) == 0 and phase is None: return cls.id(x)
        if len(x) == 1: return factory(a, b, x, phase)
        if phase is not None:  # Coherence for phase shifters.
            shift = cls.tensor(*[factory(1, 1, obj, phase) for obj in x])
            return method(cls, a, 1, x) >> shift >> method(cls, 1, b, x)
        if (a, b) in [(1, 0), (0, 1)]: # Coherence for (co)units.
            return cls.tensor(*[factory(a, b, obj) for obj in x])
        if (co, a, b) == [(True, 1, 2), (False, 2, 1)]:  # Coherence for (co)products.
            product_or_co = factory(a, b, x[0]) @ method(cls, a, b, x[1:])
            braid = x[0] @ cls.braid(x[0], x[1:]) @ x[1:]
            return product_or_co >> braid if co else braid >> product_or_co
        if a == 1:  # We can now assume b > 2.
            return method(cls, 1, b - 1, x) >> method(cls, 1, 2, x) @ (x ** (b - 2))
        if b == 1:  # We can now assume a > 2.
            return method(cls, 2, 1, x) @ (x ** (a - 2)) >> method(cls, a - 1, 1, x)
        return method(cls, a, 1, x) >> method(cls, 1, b, x)

Diagram.spiders = coherence(Spider)
Diagram.cups = nesting(lambda x, _: Spider(0, 2, x))
Diagram.caps = nesting(lambda x, _: Spider(2, 0, x))

class Functor(braided.Functor):
    def __call__(self, other):
        if isinstance(other, Spider):
            a, b, x, phase = len(other.dom), len(other.cod), other.object, other.phase
            return self.cod.ar.spiders(a, b, self(x), phase)
        return super().__call__(other)
\end{minted}
\end{python}

\begin{example}
We can now extend example~\ref{example:monoidal-formula} to arbitrary formulae of first-order logic.
Every variable that appears exactly twice is encoded as a wire (possibly with cups and caps), every variable that appears $n \neq 2$ is encoded as an $n$-legged spider.
For example, the formula $\forall c \ \forall o \ O(c, o) \land R(c) \land C(c) \implies U(c, o)$ (interpreted as ``every object of a rigid cartesian category is also its unit'') can be encoded as a diagram with a wire for $o$ and a four-legged spider for $c$.

\begin{minted}{python}
class Formula(Diagram):
    cut = lambda self: self.bubble(name="_not")

class Predicate(Box, Formula):
    def __init__(self, name, dom): Box.__init__(self, name, Ty(), dom)

def model(size: dict[Ty, int], data: dict[Predicate, list[bool]]):
    return Functor(ob=size, ar={p: [data[p]] for p in data},
                   dom=Category(Ty, Formula), cod=Category(list[int], Tensor[bool]))

objects, categories = Ty('o'), Ty('c')
has_object, has_unit = [Predicate(name, categories @ objects) for name in "OU"]
is_rigid, is_cartesian = [Predicate(name, categories) for name in "RC"]

rigid_cartesian_implies_trivial = (
    has_object >> Formula.spiders(1, 3, categories) @ objects
    >> (is_rigid @ is_cartesian @ has_unit.cut()).dagger()).cut()

size = {objects: 2, categories: 2}
predicate_values = itertools.product(*size[categories] * [[0, 1]])
relation_values = itertools.product(*size[categories] * size[o  bjects] * [[0, 1]])

for O, U, R, C in itertools.product(2 * [predicate_values] + 2 * [relation_values]):
    F = model(size, {has_object: O, has_unit: U, is_rigid: R, is_cartesian: C})
    is_rigid_cartesian_and_has_object = lambda i, j:\
        F(has_object)[i, j] and F(is_rigid)[i] and F(is_cartesian)[i]
    assert F(rigid_cartesian_implies_trivial) == all(
        not is_rigid_cartesian_and_has_object(i, j) or F(has_unit)[i, j]
        for i in range(size[categories]) for j in range(size[objects]))

rigid_cartesian_implies_trivial.draw()
\end{minted}

\ctikzfig{img/hypergraph/rigid-cartesian-implies-trivial}
\end{example}

The equality of hypergraph diagrams reduces to hypergraph isomorphism, it will be discussed in section~\ref{subsection:hypergraph-vs-premonoidal}.
The equality of non-commutative spiders is not implemented yet, spider fusion would be a natural extension of the snake removal algorithm for rigid diagrams: we find pairs fusable spiders then apply interchangers to make them adjacent.
The possible obstructions are more serious for spiders than for cups and caps however, for example consider the diagram $\spider_{0, 3}(x) \ \fcmp \ x \otimes f \otimes g \ \fcmp \ \spider_{3, 0}(x)$.
The two three-legged spiders want to fuse but the boxes $f$ and $g$ stand on the way, the best we can do is to bend their output wires with two cups and get a four-legged spider $\spider_{0, 4}(x) \ \fcmp \ x \otimes f \otimes g \otimes x \ \fcmp \ \ttcup(x) \otimes \ttcup(x)$.

\ctikzfig{img/hypergraph/obstruction}

%!TEX root = ../../THESIS.tex

\section{A premonoidal approach} \label{section:premonoidal}

\subsection{Abstract premonoidal categories}

\subsection{Concrete premonoidal categories}

\subsection{Free premonoidal categories}

\subsection{The state construction}

%!TEX root = ../../THESIS.tex

\section{Summary \& future work} \label{section:summary-and-future}

higher-dimensional diagrams


\chapter{Quantum natural language processing} \label{chapter-2:qnlp}

\section{Natural language processing with diagrams}\label{section:NLP}
\section{Classical-quantum processes with diagrams}
\section{QNLP models}
\section{Learning functors}
\section{Future work}

\chapter{Diagrammatic differentiation} \label{chapter-3:diag-diff}

\section{Dual numbers}
\section{Dual diagrams}
\section{Dual circuits}
\section{Gradients \& bubbles}
\section{Future work}

\setlength{\baselineskip}{0pt} % JEM: Single-space References

{\renewcommand*\MakeUppercase[1]{#1}%
\printbibliography[heading=bibintoc,title=References]}

\end{document}
