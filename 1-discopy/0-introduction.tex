%!TEX root = ms.tex

String diagrams are a graphical calculus for monoidal categories, introduced independently by Hotz \cite{Hotz65} in computer science and Penrose \cite{Penrose71} in physics, then formalised by Joyal and Street \cite{JoyalStreet88, JoyalStreet91}.
Graphical languages are surveyed in Selinger \cite{Selinger10},
they have become a standard tool in applied category theory.
We cite a few of the growing list of applications:
computer science \cite{BrownHutton94, Abramsky96},
quantum theory and the ZX-calculus \cite{Coecke05,AbramskyCoecke08,CoeckeDuncan08},
networks and control theory \cite{BaezErbele14, BaezFong15, BaezPollard17}, concurrency \cite{BonchiEtAl14a},
databases and
knowledge representation \cite{Patterson17, BonchiEtAl18},
Bayesian reasoning and causality \cite{CoeckeSpekkens12,ChoJacobs19,KissingerUijlen19},
linguistics and cognition \cite{ClarkEtAl08,BoltEtAl17},
functional programming \cite{Riley18},
machine learning and
game theory \cite{FongEtAl17, GhaniEtAl18}.
In all these applications, string diagrams are the syntax and structure-preserving functors are used to compute their semantics in concrete categories.

There are several existing proof assistants with graphical user interfaces for rewriting string diagrams in a more or less automated fashion:
quantomatic \cite{KissingerZamdzhiev15} and PyZX \cite{KissingervandeWetering19} for the ZX-calculus,
globular \cite{BarEtAl} and its successor homotopy.io \cite{ReutterVicary19} for higher categories,
cartographer \cite{SobocinskiEtAl19} for symmetric monoidal categories.
However, these are all stand-alone tools which use different task-specific encodings for diagrams, preventing interoperability between them and with the software ecosystems of application domains.

DisCoPy (Distributional Compositional Python) is not yet-another rewriting tool, rather it is meant as a toolbox for compiling diagrams into code, be it for neural networks, tensor computation or quantum circuits.
It provides an intuitive Python syntax for diagrams, allowing to visualise and reason about computation at a high level of abstraction.
Monoidal functors allow to translate these diagrams into concrete computation, interfacing with optimised task-specific libraries.
DisCoPy is an open source package, it is available with an extensive documentation and demonstration notebooks hosted at:

\url{https://github.com/oxford-quantum-group/discopy}

This paper describes the architecture of DisCoPy, focusing on the translation from abstract categorical definitions into their concrete implementation in Python.
We assume some working knowledge of category theory and refer the reader to \cite{Lane98} and to \cite{Awodey06} for an introduction.
Implementing a category in an object-oriented programming language amounts to defining a pair of classes for its objects and arrows, together with a pair of methods for identity and composition.
When the category is free, composition is implemented by list concatenation and identity by the empty list.
Concrete categories may then be defined by subclassing this free category and overriding identity and composition.
These are expected to respect the usual associativity and unit axioms, however they cannot be formally checked in Python.

Starting from free categories (section~\ref{1-cat}) as a base class, more structure can be added by subclassing and adding new methods.
Quotient categories can be implemented by a method for computing normal forms.
For instance, monoidal categories (section~\ref{2-monoidal}) subclass categories with an extra method for tensor product and one for interchanger normal form.
For now, we implemented Cartesian and rigid monoidal categories (section~\ref{4-rigid}), as these provide a syntax for the concrete categories implemented in DisCoPy: Python functions (appendix~\ref{3-cartesian}) and numpy \cite{VanderWaltEtAl11} tensors (section~\ref{5-tensor}).
The development of DisCoPy was first motivated by the implementation of natural language processing on quantum hardware.
Hence, we implemented quantum circuits (section~\ref{6-circuit}) as a subclass of rigid monoidal categories with an extra method for evaluation as numpy tensors and interface with the t|ket$\rangle$ compiler \cite{SivarajahEtAl20}.

We hope that this toolbox will prove to be of use to the applied category theory community, and plan on adding more categorical tools to it in the future.
