%!TEX root = main.tex

\begin{abstract}
We introduce diagrammatic differentiation for tensor calculus by generalising the
dual number construction from rigs to monoidal categories. Applying this to ZX
diagrams, we show how to calculate diagrammatically the gradient of a linear map
with respect to a phase parameter. For diagrams of parametrised quantum circuits,
we get the well-known parameter-shift rule at the basis of many variational
quantum algorithms. We then extend our method to the automatic
differentation of hybrid classical-quantum circuits, using diagrams with bubbles
to encode arbitrary non-linear operators. Moreover, diagrammatic differentiation
comes with an open-source implementation in DisCoPy, the Python library for
monoidal categories.
Diagrammatic gradients of classical-quantum circuits can then be simplified
using the PyZX library and executed on quantum hardware via the
tket compiler. This opens the door to many practical applications
harnessing both the structure of string diagrams and the computational power of
quantum machine learning.
\end{abstract}

\section*{Introduction}

String diagrams are a graphical language introduced by Penrose \cite{Penrose71}
to manipulate tensor expressions: wires represent vector spaces, nodes represent
multi-linear maps between them. In \cite{PenroseRindler84}, these diagrams are
used to describe the geometry of space-time and an extra piece of notation is
introduced: the covariant derivative is represented as a bubble around the tensor
to be differentiated. Joyal and Street \cite{JoyalStreet88,JoyalStreet91}
characterised string diagrams as the arrows of free monoidal categories, however
their geometry of tensor calculus makes no mention of differential calculus, it
only deals with composition and tensor.

In categorical quantum mechanics \cite{AbramskyCoecke08} string diagrams
are used to axiomatise quantum theory in terms of dagger compact-closed
categories. This culminated in the ZX-calculus \cite{CoeckeDuncan08},
a graphical language that provides a complete set of rules for qubit quantum
computing \cite{JeandelEtAl18a,HadzihasanovicEtAl18}. ZX diagrams
have recently been used for state-of-the-art quantum circuit optimisation
\cite{KissingerVanDeWetering20,DuncanEtAl20,DeBeaudrapEtAl20}, compilation
\cite{CowtanEtAl20,DeGriendDuncan20}, extraction \cite{BackensEtAl20} and error
correction \cite{ChancellorEtAl18,GidneyFowler19}. In recent work, ZX diagrams
have been used to study quantum machine learning \cite{Yeung20,ZhaoGao21} and
its application to quantum natural language processing
\cite{MeichanetzidisEtAl20a,CoeckeEtAl20}.

In this work, we introduce diagrammatic differentiation: a graphical notation
for manipulating tensor derivatives. On the theoretical side, we generalise
the dual number construction (discussed in section~\ref{1-dual-numbers})
from rigs to monoidal categories (section~\ref{2-dual-diagrams}). We then apply
this construction to the category of ZX diagrams (section~\ref{2b-differentiating-zx})
and of quantum circuits (section~\ref{3-dual-circuits}). In section~\ref{4-bubbles}
we give a formal definition of diagrams with bubbles and their gradient with the
chain rule. We use this to differentiate quantum circuits with neural
networks as classical post-processing. The theory comes with an
implementation in DisCoPy \cite{DeFeliceEtAl20}, the Python library for
monoidal categories. The gradients of classical-quantum circuits can then
be simplified using the PyZX library \cite{KissingerVanDeWetering19} and compiled
on quantum hardware via the tket compiler \cite{SivarajahEtAl20}.

\section*{Related work}

The same bubble notation for vector calculus is proposed in \cite{KimEtAl20},
but they have mainly pedagogical motivations and restrict themselves to the case
of three-dimensional Euclidean space. To the best of our knowledge, our
definition is the first formal account of string diagrams with bubbles for
tensor derivatives.

Differential categories \cite{BluteEtAl06} have been introduced to axiomatise
the notion of derivative. More recently reverse derivative categories
\cite{CockettEtAl19} generalised the notion of back-propagation, they have been
proposed as a categorical foundation for gradient-based learning
\cite{CruttwellEtAl21}. These frameworks all define the derivative
of a morphism with respect to its domain. In our setup however, we define
the derivative of parametrised morphism with respect to parameters that are
in some sense external to the category. Investigating the relationship between
these two definitions is left to future work.
